\chapter{Deep Reinforcement Learning}\label{chap:drl}

This chapter consider another important application of machine learning to robotics, i.e., the utilisation of deep reinforcement learning agent for robot motion planning and control. 
%
We will first present some preliminaries on training DRL policy in robotics, followed by the discussion on the sample efficiency concerning the sufficiency of training (Section~\ref{sec:drlsufficiencytraining}) and the introduction of several statistical methods for evaluation (Section~\ref{sec:DRLevaluation}). Afterwards, we will discuss how to formally express the properties (Section~\ref{sec:DRLproperties}) and then focus on reusing the verification tools  for convolutional neural network to work with deep reinforcement learning, by considering the verification of policy generalisation (Section~\ref{sec:DRLverification}), the verification of state-based policy robustness (Section~\ref{sec:DRLRobustnessverification}), and the verification of temporal policy robustness (Section~\ref{sec:verificationtemporalDRLrobustness}). In addition, we will discuss how to address the well-known Sim-to-Real challenge in robotics with the verification techniques (Section~\ref{sec:verificationsim2real}).  


%
Unlike the convolutional neural network 
%we introduced in Chapter~\ref{chap:perception}
which relies on labelled data for supervised learning, reinforcement learning is a learning process in which the learning agent is trained through trial and error. Reinforcement learning is usually formalised as the finding of an optimal strategy in a Markov decision process (MDP). In other words, it is typical to model the interaction of an intelligent agent with its environment as an MDP, and then apply some algorithms (e.g., value iteration, policy iteration) to compute an optimal strategy for the intelligent agent. 
%
To utilise traditional MDP algorithm, it is needed to have several components well defined, including the states, the actions, the transition relation, and the reward function. However, for a real-world application, such components may not be easily defined, for example, the transition relation may not be definable as it can be impossible to have a complete definition of the environment. Moreover, some components such as the states may be of very high dimensional, which will make the traditional MDP algorithms fail to work due to the time and memory limitations. To deal with these problems, deep reinforcement learning (DRL) combines reinforcement learning and deep learning to enable our working with unstructured, high-dimensional data without manual engineering of the components. 
%
Another major difference from the convolutional neural network is the definition of safety properties. While for convolutional neural network the safety properties are closely related to the misclassification of individual input instances, for reinforcement learning the safety properties are more appropriate to be defined on the \emph{sequential inputs}.
%, i.e., the executions of the interaction from initial states to termination states. 



We remark that, in this chapter, we assume that the reward function is well defined, and it is based on this that we consider the safety properties. Other important factors related to the definition of rewards, such as the side effect and reward hacking \cite{DBLP:journals/corr/AmodeiOSCSM16}, are not considered. 

\section{Interaction of Agent with Environment}\label{sec:mdp}

%\gls{DRL} algorithms have been widely used in robotics applications thanks to its ability to learn intelligent agents in complex environments. \gls{DRL} algorithms are concerned with how the robot should act in an environment, and decide a strategy to maximise the trajectory rewards. Therefore, the

We use discounted infinite-horizon MDP to model the interaction of an agent with the environment $E$. An MDP is a 5-tuple ${\cal M}^E=(\mathcal{S},\mathcal{A}, \mathcal{P}, \mathcal{R}, \gamma)$, 
where $\mathcal{S}$ is the state space, $\mathcal{A}$ is the action space, $\mathcal{P}(\textbf{x}'|\textbf{x},\textbf{a})$ is a probabilistic  transition, $\mathcal{R}(\textbf{x},\textbf{a})\in {\mathbb R}_{\ge 0}$ is a reward function, 
and $\gamma\in [0,1)$ is a discount factor. We use $\textbf{x}$ to range over the state space $\mathcal{S}$ because it  not only is a state but also will later be used as input to a policy neural network.  
We consider DDPG  \cite{lillicrap2015continuous,sutton2018reinforcement,mnih2013playing} for a reinforcement learning algorithm, although there are many other deep reinforcement learning algorithms. DDPG returns a deterministic policy. 
A (deterministic) policy $\pi$ includes a mapping $\mu: \mathcal{S} \rightarrow \mathcal{A}$ that maps from states to actions.

Based on $\mathcal{M}^E$, a policy $\pi$ induces a trajectory distribution $\rho^{\pi,E}(\zeta)$ where 
\begin{equation}
    \zeta=(\textbf{x}_0,\textbf{a}_0,\textbf{x}_1,\textbf{a}_1,...)
\end{equation} denotes a random trajectory. The state-action value function of $\pi$ is defined as 
\begin{equation}
    Q^{\pi}(\textbf{x},\textbf{a})= \mathbb{E}_{\zeta\sim \rho^{\pi,E}}[\sum_{t=0}^{\infty}\gamma^t\mathcal{R}(\textbf{x}_t,\textbf{a}_t)]
\end{equation} and the state value function of $\pi$ is 
\begin{equation}
    V^\pi(\textbf{x})=Q^{\pi}(\textbf{x},\pi(\textbf{x})). 
\end{equation} %We write $\pi^*$ for the optimal policy and $Q^{*}$ and $V^*$ as its respective value functions. 
%In Section~\ref{sec.3}, we will explain how to construct a DTMC to approximate $\rho^{\pi,E}(\zeta)$. 


\iffalse

%\gls{DRL} problem can be formulated as a \gls{MDP}, which includes the state space $\mathcal{S}$, action space $\mathcal{A}$, transition probability function $\mathcal{P}$, reward function $\mathcal{R}$ and discount factor $\gamma$ in detail.

In terms of the target policy $\pi:\mathcal{S}\rightarrow\mathcal{A}$, a value function $V^\pi$ is designed as a description of total discounted reward $G_t$ for each state $s\in\mathcal{S}$:
\begin{equation}
    V^\pi(s) = \mathbb{E}_\pi[G_t|\textbf{x}_t=s]
\end{equation}
where $t$ is the discrete time step and $G_t$ is the expected return after time step $t$.

% \xiaowei{$t$ is not defined. the left hand side $V^\pi(s)$ has no $t$ but the right hand side has, why? }

With the Bellman equation \cite{bellman1966dynamic}, $V^\pi$ can be represented by a recursive form:
\begin{equation}
    V^\pi(s) = \mathbb{E}_\pi[r_t+\gamma V^\pi(\textbf{x}_{t+1})|\textbf{x}_t=s]
\end{equation}

The action-value function $Q^\pi$ is formulated as follows:
\begin{equation}
    Q^\pi(\textbf{x},\textbf{a}) = \mathbb{E}_\pi[r_t+\gamma Q^\pi(\textbf{x}_{t+1},\textbf{a}_{t+1})|\textbf{x}_t=s,\textbf{a}_t = a]
\end{equation}
The \gls{DRL} algorithms are trying to find an optimal action that maximises the action-value function or the value function: $\pi^*(s) = \arg\max_{a\in\mathcal{A}} Q^*(\textbf{x},\textbf{a})$, here $Q^*(\textbf{x},\textbf{a})$ is the value of taking action $\textbf{a}$ in state $s$ under the optimal policy $\pi^*$. 
% \xiaowei{should be $\pi^*(s)$? also, $Q^*$ is not defined. }

\fi

\begin{example}\label{example:robotnavigation}
We consider 
%an autonomous 
a reinforcement learning driven robot 
%system 
%to auto navigation
that 
%automatically 
navigates, and avoids collisions, in a complex environment where there are static and dynamic objects (or obstacles).  %
%, due to its unlikely to learn models on all possible environments, and 
%in some cases it 
%with such a complex environment, the autonomous robot 
%is impractical to be trained in a real-world environment where the negative examples are of high costs. 
%with no priority knowledge 
%due to the high cost of negative examples. 
%
%Moreover, the studied autonomous robot system has continuous action sets and continuous state sets, which are hard to formulated with a discrete algorithm. Thus, a model-free, off-policy algorithm, named \gls{DDPG}, is applied in this paper. 
%
%To compatible with the \gls{DDPG} algorithm, 
%As stated in Section~\ref{sec:mdp}, 
The interaction of robot with the environment 
%autonomous 
%system 
%has been 
can be modelled as an MDP. At each time $t$, %the 
%autonomous 
%robot 
the robot has its observation of the laser sensors from the environment, namely state $\textbf{x}_t$, i.e.,  
%set as:
	\begin{equation}\label{states}
		\textbf{x}_t = (o^1_t,o^2_t,\cdots,o^n_t)^T
	\end{equation}
where 
%$\textbf{x}_t\in \mathcal{S}$ presents the observable information and 
$o^1_t,o^2_t,\cdots,o^n_t$ are 
%detailed 
sensor signals at time $t$. 
%$\mathcal{S}$ is state-space or all possible states of the robot in the environment. 
The sensors can only observe partial information of the environment, e.g., by scanning the environment within a certain distance. For example, the observation range is within 3.15 metres in Turtlebot Waffle Pi \cite{name} for a distance sensor. 

%The 
An action $\textbf{a}_t\in\mathcal{A}$ consists of several decision variables.
%made by actor networks. 
With the PID controller on the 
%autonomous 
robot, %the actor networks 
%it only needs to decides 
we consider two  action variables, representing line velocity and angle velocity, respectively, i.e., 
\begin{equation}
    \textbf{a}_t = (v^{line}_t, v^{angle}_t)^T.
\end{equation}
%where $v^{line}_t, v^{angle}_t$ are the line velocity and angle velocity, respectively. Here,  $\mathcal{A}$ is the set of all possible moves that the robot can make. 
At each time $t$, the DRL policy outputs an action $\textbf{a}_t$ from the action set $\mathcal{A}$.


The objective of the robot is to avoid the obstacles and reach a goal area. On every state $\textbf{x}_t$, the sensory input $o^i_t$ can be utilised to e.g., predict the distance to the obstacles and the goal area when they are close enough (within 3.15 metres). To implement the objective, the environment may impose a reward function $\mathcal{R}$ on the states or the actions or both. A reward on the states can be e.g., with respect to the distance to obstacles, and a reward on the actions can be e.g., with respect to the acceleration in linear or angular speed. 
\end{example}





\section{Training a Reinforcement Agent}

The model-free
%, off-policy 
DDPG algorithm \cite{lillicrap2015continuous} is applied for the training of a DRL policy for the robot. Typically, the DRL policies are 
%always 
trained in a simulation environment before applied to the real world \cite{christiano2016transfer}. That is because of the unbearable costs of having real-world (negative) examples for training in real world \cite{yu2018towards}. 
The objective of the DDPG algorithm for an intelligent agent is to learn a policy $\pi$, which maximises the expectation of the best reward over time $N$:
	\begin{equation}
		\mathcal{J} = \mathbb{E}(\sum_{t=0}^{N-1}\gamma^tr_{t}| \textbf{x}_t = \textbf{x}_0)
	\end{equation}
where $\textbf{x}_0$ is the initial state; $r_t$ is the reward based on state-action pair at time slot $t$; $\gamma$ is the discount factor which is applied to reduce the effect of future reward. To simplify the notation, we let $G_t = \sum_{t=0}^{N-1}\gamma^tr_{t}$.

\begin{figure}[htbp]
\centerline{\includegraphics[width=7cm]{images/LookFurther/DDPG.pdf}}
\caption{Structure of DDPG.}
\label{ddpg}
\end{figure}

The DDPG algorithm has two different neural networks, actor networks $\mu(\textbf{x}_t|\theta^\mu)$ and critic networks $\nu(\textbf{x}_t,\textbf{a}_t|\theta^\nu)$, which are illustrated in Fig. \ref{ddpg}. $\theta^\mu$ and $\theta^\nu$ are the weights of the actor and critic network, respectively. Due to the non-linearity of the neural networks, the DDPG algorithm can deal with the continues states and continues actions, which are more realistic to autonomous systems. Actor network is used to yield a deterministic action value $\textbf{a}_t$, it takes the observation of environment as the inputs, and outputs the decided actions of the robot. Critic network is used to approximate $Q$ value, which is use to determine whether the state-action pair is good or not. It takes the observation from environment and the action value from actor networks, and then outputs the approximated $\nu(\textbf{x}_t,\textbf{a}_t)$ values. In the algorithm, the 
critic network is trained to minimise the loss function based on the stochastic gradient descent \cite{lillicrap2015continuous}:
\begin{equation}
    \mathcal{L}(\theta^\nu) = \mathbb{E}[(y_t - \nu(\textbf{x}_t,\textbf{a}_t|\theta^\nu))^2]
\end{equation}
where $y_t = r_t(\textbf{x}_t,\textbf{a}_t)+\gamma \nu(\textbf{x}_{t+1},\mu(\textbf{x}_t|\theta^\mu)|\theta^\nu)$ is the approximated $Q$ value based on current state $\textbf{x}_t$ and previous parameters $\theta^\mu, \theta^\nu$ of two neural networks.
%
The actor network is updated by the policy gradient with following equation \cite{lillicrap2015continuous}:
\begin{align}\label{DDPG}
    \nabla_{\theta^\mu}\mathcal{J}^{\theta^\mu} &= \mathbb{E}[\nabla_{\textbf{a}}\nu(\textbf{x},\textbf{a}|\theta^\nu)|_{\textbf{a}=\mu(\textbf{x}|\theta^\mu)}\nabla_{\theta^\mu}\mu(\textbf{x}|\theta^\mu)]
\end{align}

For the training, to breaking harmful correlations and learn from individual tuples for  multiple times, an experience replay buffer is applied. 
%Also, two additional neural networks, $\mu'(s|\theta^{\mu'})$ and $Q'(\textbf{x},\textbf{a}|\theta^{Q'})$, named target actor networks and target critic networks, are introduced to stabilize the training procedure \cite{van2016deep}.

For DDPG, a learned policy $\pi$ includes both actor network $\mu$ and critic network $\nu$. For some other DRL algorithms such as DQN, a learned policy $\pi$ may include only a  network $\mu$. 


\section{Sufficiency of Training Data}\label{sec:drlsufficiencytraining}

Theoretically, it is shown in \cite{DBLP:conf/alt/WeiszAS21} that the sample complexity, i.e., the number of  training-samples needed in order to successfully train a good model, is exponential with respect to either the input dimension or the finite horizon (of the sampled episodes). However, practical cases might not be as pessimistic as this looks like. 


Empirically, we can adapt the learning curve idea (which plots the test accuracy with respect to the training set size) to this context, and plot the curve to show the achieved performance as the function of the number of episodes played during the learning. Actually, the number of episodes played during the learning is a concept similar as the size of training dataset for the supervised training of classifiers, and we can define the achieved performance as e.g., expected reward over a number of rollouts. By tracing the curve, we will be able to know when additional episodes will not make a significant change to the training result. 


\section{Statistical Evaluation Methods}\label{sec:DRLevaluation}

Before introducing verification techniques which are usually of high computational complexity and hence are more suitable for safety-critical applications, we need some low-complexity evaluation methods to understand roughly how well a DRL algorithm works during training  and how well a trained DRL policy works in an environment. Those model evaluation methods for general machine learning models (such as ROC and PR curves) can still be applicable, but in this section we will introduce a few DRL specific evaluation methods. 



The below methods are all based on certain random variable $X$, and to track the trend and variance of $X$ in the training or test phase. 


\subsection*{Evaluation of DRL Training Algorithm}



In the training phase, the random variable $X$ can be e.g., per-epoch reward. Given a set of $X$'s values, they form a distribution. It is useful to use the dispersion (also called variability, scatter, or spread) of this distribution to understand if the training process goes well. The measurement of the dispersion of a distribution can be with a few indicators \cite{DBLP:journals/corr/abs-1912-05663} such as 
\begin{itemize}
    \item Interquartile Range (IQR). A distribution can be divided into 100 percentiles, denoted as $Q_1$, ..., $Q_{100}$, respectively. IQR can be defined as the difference between e.g., the 25th and 75th  percentiles, i.e., 
\begin{equation}
    \text{IQR}(X) = Q_{75}(X) - Q_{25}(X)
\end{equation}
    \item Conditional Value at Risk (CVaR). CVaR concerns the worst cases, by taking a weighted average of the worst-case losses in the left-most tail of the distribution of possible outcomes, formally 
    \begin{equation}
        \text{CVaR}_\alpha(X) = \mathbb{E}[X~|~X\leq \text{VaR}_\alpha(X)]
    \end{equation}
    where $\text{VaR}_\alpha$ (Value at Risk) is just the $\alpha$-th quantile (e.g., quartile, percentile, or decile) of the distribution of $X$.
\end{itemize}

Then, we may consider the change of the above indicators over the training time in a single training run or across different training runs. For a single training run, we can split it into a number of sliding widows and then collect $X$'s values from the sliding windows. When working with a set of different training runs, we can collect $X$'s values from those runs. 



\subsection*{Evaluation of DRL model}

Once there is a trained DRL policy, we can track its performance over a rollout by plotting the cumulative reward as a function of the number of steps. On such a plot, we may concern e.g., the slope, the minimum, and the zero-crossing (a point where the sign of the curve changes). 

Moreover, we can also consider $X$ as the performance over a set of rollouts, and use the IQR and CVaR to understand the dispersion of the distribution of $X$. 




\section{Safety Properties through Probabilistic Computational Tree Logic}\label{sec:DRLproperties}

%This section considers how to determine ${\cal M}^E(\pi,\textbf{x}_0)\models \phi$ when given a model ${\cal M}^E(\pi,\textbf{x}_0)$ and a property $\phi$. 

Probabilistic model checking \cite{kwiatkowska_probabilistic_2018} has been
used to analyse quantitative properties of systems across a variety of application domains. It involves the construction of a probabilistic model, e.g., DTMC or MDP, that formally represents the behaviour of a system over time. The properties of interest are usually specified with, e.g., LTL or PCTL. Then, via model checkers, a systematic exploration and analysis is performed to check if a claimed property holds. In this section, we adopt DTMC and PCTL whose definitions are as follows. %We remark that, the application of a policy $\pi$ over an MDP ${\cal M}^E$ induces a DTMC ${\cal M}^E(\pi,\textbf{x}_0)$. 

\begin{definition}[DTMC]
Let $AP$ be a set of atomic propositions. A DTMC is a tuple $(S,\textbf{x}_0,\textbf{P},L)$, where 
%\begin{itemize}
%\item 
$S$ is a (finite) set of states, $\textbf{x}_0\in S$ is an initial state, 
%\item 
$\textbf{P}:S\times S \rightarrow [0,1]$ is a probabilistic transition matrix such that $\sum_{s^{\prime}\in S}\textbf{P}(s,s^\prime)=1$ for all $s\in S$, and 
%\item 
$L:S\rightarrow 2^{AP}$ is a labelling function assigning  each state with a set of atomic propositions.
%from $AP$.
%\end{itemize}
\end{definition}
\begin{definition}[DTMC Reward Structure]
A reward structure for DTMC $D=(S,\textbf{x}_0,\textbf{P},L)$ is a tuple $r=(r_S, r_T)$ where $r_S:S\rightarrow \mathbb{R}_{\ge 0}$ is a state reward function and $r_T:S\times S \rightarrow \mathbb{R}_{\ge 0}$ is a transition reward function.
\end{definition}




\begin{definition}[PCTL]
The syntax of PCTL is defined by \emph{state formulae} $\phi$, \emph{path formulae} $\psi$ and \emph{reward formulae} $\mu$.
\begin{equation}\nonumber
\begin{aligned}
\phi &::= true \mid ap \mid \phi \wedge \phi \mid \neg \phi \mid {P}_{\bowtie p}(\psi) \mid {R}_{\bowtie q}^{r}(\mu)
\\
\psi &::= \next \: \phi \mid 
%\phi \: U^{\leq t} \: \phi \mid 
\phi \: U \: \phi 
\\
\mu &::= 
%I^{=t} \mid
C^{\leq t} \mid \eventually \: \phi
\end{aligned}
\end{equation}
where $ap \in AP, p\in [0,1], q\!\in \!\mathbb{R}_{\geq 0}, t\!\in\! \mathbb{N}$, $\bowtie \in \{<,\leq,>,\geq\}$ and $r$ is a reward structure. 
%
The temporal operator $\next$ is called ``next'', and  %$U^{\leq t}$ is called ``bounded until'' while 
$U$ is called ``until''. 
We write $\eventually \, \phi$ 
%as a syntax sugar 
for $true \, U \, \phi $, and call it ``eventually''. Operator 
%$I^{=t}$ represents the state reward at time step $t$---``instantaneous reward'', while 
$C^{\leq t}$ is ``bounded cumulative reward'', expressing the reward accumulated over $t$ steps.
%(when $t=\infty$, we write $C$ for short as the total reward accumulated indefinitely). 
Formula ${R}_{\bowtie q}^{r}(\eventually \, \phi)$ expresses ``reachability reward'', the reward accumulated up until the first time a state satisfying $\phi$. %\xiaowei{two $F$ formulas. to diamond}


%State formula $\phi$ is evaluated 
%to be either true or false in 
%on states.
%each state. 
Given $D=(S,\textbf{x}_0,\textbf{P},L)$ and $r=(r_S, r_T)$, the satisfaction of state formula $\phi$ on a state $s\in S$ is defined as:
%\begin{equation}
\begin{align}
s & \models true; \quad
s \models ap \, \Leftrightarrow\, ap \in L(s);\quad
s  \models \neg \phi \,\Leftrightarrow\, s \not\models \phi ; \nonumber
\\
s & \models \phi_1\wedge\phi_2 \, \Leftrightarrow\, s \models \phi_1 \text{ and } s \models \phi_2; \nonumber
\\
s & \models \mathcal{P}_{\bowtie p}(\psi) \,\Leftrightarrow\, Pr(s\models \psi)\bowtie p ; \nonumber
\\
s & \models \mathcal{R}_{\bowtie q}^{r}(\mu) \,\Leftrightarrow\, \mathbb{E}[rew^{r}(\mu)] \bowtie q, \nonumber
\end{align}
%\end{equation}

\noindent where $Pr(s\models \psi)\bowtie p $ concerns the probability of the set of paths that satisfy $\psi$ and start in $s$. Given a path $\eta$, if write $\eta[i]$ for its \textit{i}-th state and $\eta[0]$ the initial state, then
%
{\small
\begin{align}
%rew^{r}(I^{=t})(\eta)&=r_S(\eta[t]) \nonumber
%\\
rew^{r}(C^{\leq t})(\eta)&= \sum_{j=0}^{k-1}(r_S(\eta[j])+r_T(\eta[j],\eta[j+1])) \nonumber
\\
rew^{r}(\eventually\phi)(\eta)&=\begin{cases} \infty & \forall j \in \mathbb{N}(\eta[j] \not\models \phi) \\ rew^{r}(C^{\leq ind(\eta,\phi)})(\eta) & \text{otherwise}\end{cases} \nonumber
\end{align}}\normalsize
where $ind(\eta,\phi)=\min\{j|\eta[j] \models \phi\}$ denotes the index of the first occurrence of $\phi$ on path $\eta$.
Moreover, the satisfaction relations for a path formula $\psi$ on a path $\eta$ is defined as:
\begin{equation}\nonumber
\begin{aligned}
\eta & \models \next\phi \,\Leftrightarrow\, \eta[1] \models \phi 
\\
%\eta & \models \phi_1 \, U^{\leq t}\,\phi_2 \,\Leftrightarrow\,  \exists 0 \leq j \leq t \nonumber
%\\
% & \quad(\eta[j]\models \phi_2\wedge(\forall 0\leq k<j \; \eta[k]\models \phi_1))\\
\eta & \models \phi_1 \, U\,\phi_2 \,\Leftrightarrow\,  \exists j \geq 0 (\eta[j]\models \phi_2\wedge \forall k<j(\eta[k]\models \phi_1))
\end{aligned}
\end{equation}
\end{definition}

Very often, it is of interest to know the actual probability that a path formula is satisfied, rather than just whether or not the probability meets a required threshold since this can provide a notion of margins as well as benchmarks for comparisons following later updates. So, the \gls{PCTL} definition can be extended to allow \textit{numerical queries} of the form $\mathcal{P}_{=?}(\psi)$ or $\mathcal{R}^r_{=?}(\psi)$ \cite{kwiatkowska_probabilistic_2018}.
After formalising the system behaviors and properties in {DTMC} and {PCTL} respectively, 
%the verification focuses on checking of \textit{reachability} in a \gls{DTMC}. 
%That is, \gls{PCTL} expresses the constraints that must be satisfied, concerning the probability of, starting from the initial state, reaching some states labelled as, e.g., unsafe and success 
automated tools have been developed to solve the verification problem, e.g., PRISM \cite{kwiatkowska_prism_2011} and STORM \cite{dehnert_storm_2017}.
%\footnote{Since each tool of its current version (at the time of writing) has some limitations, e.g., PRISM cannot check conditional probabilities while STORM cannot check \gls{LTL}-style properties.}.
%which employ a symbolic model checking algorithm to calculate the probability that a path formulae is satisfied. 
%
We remark that, PCTL can be utilised to describe safety-related properties for e.g., the robot navigation example  Example~\ref{example:robotnavigation} as discussed in \cite{DBLP:journals/corr/abs-2109-06523}. 


In the following sections, we will discuss several instantiations of the DTMC  as ${\cal M}^E(\pi,\textbf{x}_0)$ (Section~\ref{sec:DRLverification}), ${\cal M}^E(\pi,\textbf{x}_0,\mathcal{C})$ (Section~\ref{sec:DRLRobustnessverification}), ${\cal M}^E(\pi,\mathcal{C}(\textbf{x}_0))$ (Section~\ref{sec:verificationtemporalDRLrobustness}), and  ${\cal M}^{E_1\times E_2}(\pi,(\textbf{x}_0,\textbf{x}_0))$ (Section~\ref{sec:verificationsim2real}), respectively. 

%It is worth noting that both \gls{DTMC} and \gls{PCTL} can be augmented with rewards/costs \cite{filieri_probabilistic_2013}, which can be used to model, e.g. the energy consumption of \gls{RAS} \cite{zhao_toward\textbf{x}_2019}. As readers will see, we also utilise rewards/costs in formalising some of our properties, e.g., resilience.

\begin{comment}%% If we have the space to present the actually PRISM model, we then need to insert back the following description.
In general, a PRISM module contains a number of local variables which constitute the state of the module. The transition behaviour of the states in a module is described by a set of commands which take the form of:
\begin{equation}
    [Action] \: Guard \rightarrow Prob_1 : Update_1 + ... + Prob_n : Update_n ; \nonumber
\end{equation}
As described by the PRISM manual\footnote{https://www.prismmodelchecker.org/manual/}, the guard is a predicate over all the variables (including those belonging to other modules. Thus, together with the action labels, it allows modules to synchronise). Each update describes a transition which the module can make if the guard is true. A transition is specified by giving the new values of the variables in the module, possibly as a function of other variables. Each update is also assigned a probability (in our \gls{DTMC} case) which will be assigned to the corresponding transition. 
\end{comment}



\section{Verification of Policy Generalisation}\label{sec:DRLverification}


Although at any specific time a DRL agent with actor network $\mu$ -- like the classifier -- also returns an action $\mu(\textbf{x})$ according to the state $\textbf{x}$, the correctness of the action $\mu(\textbf{x})$ is not solely dependent on the state $\textbf{x}$. Instead, considering its training mechanism, the correctness of the action at any specific time depends on the expected long-term accumulated rewards. For this reason, the verification of a DRL agent also needs to consider not only the current state but also the long-term rewards (and therefore the future states). Therefore, to understand if a learned policy $\pi$ works well in an environment MDP ${\cal M}^E$, we can conduct probabilistic model checking on their induced DTMC ${\cal M}^E(\pi,\textbf{x}_0)$, where $\textbf{x}_0$ is an initial state of ${\cal M}^E$. This enables the analysis of various properties that can be expressed with PCTL. 

When the exact construction of the DTMC ${\cal M}^E(\pi,\textbf{x}_0)$ is hard (due to e.g., some components of the environment ${\cal M}^E$ is unknown, or the policy network is too big for analysis), as discussed in \cite{DBLP:journals/corr/abs-2109-06523}, we can approximate it from a set of sampled trajectories. 

Actually, due to the high-dimensionality of the underlying control problem and the continuity of some state and observable variables, it is unlikely that we can construct a DTMC that is exactly the application of policy $\pi$ to MDP ${\cal M}^E$. Certain abstraction techniques \cite{clarkebook} will be needed, for example, we consider predicate abstraction, where the DTMC's state space is constructed from a given set of predicates over MDP's  state variables, as we will discuss below. 

\subsection*{Construction of a DTMC Describing the Failure Process}\label{sec:DTMCconstruction}

We consider the execution of the policy $\pi$ in an environment. For simplicity, we only differentiate the environments with a disturbance level that the robot's sensory input may be subject to, and assume that the disturbance level follows a distribution $\mathcal{N}(0,\sigma)$.
Now, as stated in Section~\ref{sec:mdp}, given an MDP ${\cal M}^\sigma$ (based on a disturbance $\mathcal{N}(0,\sigma)$) and a DRL policy $\pi$, there is a trajectory distribution $\rho^{\pi,\sigma}(\zeta)$.  
Based on the \textit{dynamics of risk-levels} in $\rho^{\pi,\sigma}(\zeta)$,  
%during the \gls{RAS} mission, 
we 
%first 
can define a DTMC, 
%structure 
%for the proposed assessment framework, 
as shown in Fig. \ref{dtmc}. It consists of a ``negligible-risk'' state $s_N$, a catastrophic failure state $s_C$, and several states $s_{B_i}$ representing different levels of ``benign failures''.
%(risky states that might lead to a catastrophic failure later). 
%TODO: remove all \vspace once accepted...
%\vspace{-1mm}
\begin{figure}[htbp]
\centerline{\includegraphics[width=0.4\textwidth]{images/LookFurther/DTMC.pdf}}
\centering
%\vspace{-2mm}
\caption{The failure process DTMC based on risk-levels.}
\label{dtmc}
\end{figure}%\vspace{-2mm}

%Given a trained
%/frozen 
%\gls{DRL} policy, statistical testing is conducted to test the \gls{RAS} with different disturbance-levels (representing noisy environmental factors), yielding a set of \textit{mission trajectories}. Note, each
%We can use statistical testing to get a finite set of trajectories from $\rho^{\pi,\sigma}(\zeta)$. 
Each trajectory is a sequence of successive states %(or path) 
from the initial state to the end state of a DRL episode. First, we map each state in the trajectories to one of the states describing the failure process (i.e., $s_N$, $s_C$, and $s_{B_i}$). Second, we may conduct statistical analysis on the frequency of transitions between $s_N$, $s_C$, and $s_{B_i}$, based on which we 
%invoke estimators to 
estimate their corresponding transition probabilities. Finally, we construct the failure process DTMC with the defined structure and the estimated transition probabilities.
%(which is encoded later by some formal language and fed into model checkers). 
To be exact, we describe the 3 main steps above as what follows.
%The \gls{DRL} policy is abstracted into the induced DTMC, and the normal and safe states are modelled in the non-risky route state. In the proposed DTMC framework, each state can transit into one or more unsafe states with different probabilities, e.g. in a safe driving state $s_G$, the autonomous robot may fall into an unsafe state $B_3$ if there is a suddenly injected obstacle. We divide the unsafe states into several different levels, $B_1, B_2, \cdots, B_n$ to reveal different failure levels. The far from the safe states, the higher the possibility of mission failure, but none of the benign failures will cause any catastrophic damage to the autonomous robot.



\subsubsection{Mapping MDP States onto DTMC States}

% 
%Without loss of generality, 
%in this paper, 
First of all, every state in the DTMC (cf. Fig.~\ref{dtmc}) is associated with a risk level. Specifically, $s_N$ is the negligible-risk state, $s_C$ is the catastrophic
failure state, and $s_{B_i}$ are benign failure states such that the risk on $s_{B_i}$ is higher than on $s_{B_j}$ if $i>j$. 

Now, to map $\mathcal{S}$ (the states on the trajectories) onto $S$ (the states on the DTMC), we define a measure of risk based on the distance of the robot to obstacles. 
%
For instance, $s_N$ suggests that the robot is 3+ metres away from the obstacle, $s_{B_1}$ suggests 2-3 metres away, 
$s_{B_2}$ suggests 1-2 metres away, etc.  %and  
%but within 3m 
%represents state  
%(similarly for $s_{B2}$, etc), 
%and.
Moreover, catastrophic failure $s_C$ is defined as the robot terminated unexpectedly by a non-recoverable failure.  The determination of the risk levels for states in $\mathcal{S}$ can be done by evaluating the sensory input. We remark that, this is a predicate abstraction where $s_N$, $s_{B_1}$, $s_{B_2}$, and $s_C$ can be seem as predicates such that only one predicate can be True on any MDP state. 

%for every state $s\in \mathcal{S}$, we have a measure of risk. W.l.o.g.,
%our measure of risks
%we define the measure based on the distance to obstacles. 

\iffalse

\begin{definition}[Negligible-Risk State]
A state $s$ is mapped onto the negligible-risk state $s_N$  if the measure of risk on $s$ is below a pre-specified safe threshold.
%is defined as an .
\end{definition}
\begin{definition}[Benign Failure State]
A state, denoted as $s_{B_i}$, in which the measure of risk is above the safe threshold but lower than a specified level $B_i$ (while the \gls{RAS} mission is still ongoing) is defined as a benign failure state.
\end{definition}
\begin{definition}[Catastrophic Failure State]
A state, denoted as $s_{C}$, in which we observe the \gls{RAS} mission is terminated unexpectedly by a non-recoverable catastrophic failure is defined as a catastrophic failure state.
\end{definition}
\fi

%\noindent 
\begin{definition}[Negligible-Risk Route]
Given an MDP ${\cal M}^\sigma$ and a DRL policy $\pi$, a \textit{negligible-risk route} is defined as a mission trajectory in $\rho^{\pi,\sigma}(\zeta)$ that contains only $s_N$ states.
%the non-risky route in a \gls{RAS} environment. The \textit{Non-Risky Route} is the safest route, but it may not be the route with the highest reward under the concept of \gls{DRL}. \xiaowei{I believe we need a formal definition on the ``non-risky route''. Given a policy, there might not exist a path which stays on $s_G$. }
\end{definition}

 We remark that,
\iffalse
: (i) Although in some extreme 
cases 
%environments
with high-level of disturbance the negligible-risk route may not exist in practice, there is always a negligible-risk route in theory (potentially with 
%extreme 
small probabilities); (ii) The 
\fi
the negligible-risk route is not necessarily the optimal route achieving the highest reward,
%in the training of the DRL, 
rather it 
only depends on the 
%observations of the 
risk-levels during the RAS mission.



\subsubsection{Estimating Transition Probabilities}

We can collect a set of mission trajectories by conducting 
%After the 
statistical testing (the simple Monte Carlo sampling in our case) on $\rho^{\pi,\sigma}(\zeta)$. 
%Given a trajectory, mapping each of its state to one of the states describing the failure process will result a sequence of states consisted of $s_N$, $s_C$ and $s_{B_i}$, and thus transitions among them as well. 
Then, all mission trajectories collectively can be transformed into a 
%large 
set of transitions, based on which we build a transition matrix to record the statistical data as follows:
\begin{table}[h!]
\centering
\begin{tabular}{l|lllll}
         & $s_N$       & $s_{B_1}$  & ... & $s_{B_m}$      & $s_C$         \\ \hline
$s_N$    & $n_{1,1}$   & $n_{1,2}$ & ... & $n_{1,m+1}$   & $n_{1,m+2}$   \\
$s_{B_1}$ & $n_{2,1}$   & $n_{2,2}$ & ... & ...           & ...           \\
...      & ...         & ...       & ... & ...           & ...           \\
$s_{B_m}$ & $n_{m+1,1}$ & ...       & ... & $n_{m+1,m+1}$ & ...           \\
$s_C$    & $n_{m+2,1}$ & ...       & ... & ...           & $n_{m+2,m+2}$
\end{tabular}
\end{table}
\newline
where $n_{1,1}$ records the number of transitions from $s_N$ to $s_N$, and so on.
%so forth. 
%While 
$m$ is the number of levels of benign failures (that varies case by case depends on the application-specific context, e.g., we choose $m=3$ in our 
%later 
experiments).

%Let us denote 
Let the transition probability matrix of the failure process DTMC be $\textbf{P}_1=(p_{ij})\in [0,1]^{(m+2)\times (m+2)}$. 
In a DTMC, given a current state $i$, the transition to a next state follows a \textit{categorical distribution}. Due to the Markov property, the categorical distributions of each state are \textit{independent}. Hence, as we observe repeated outgoing transitions from state $i$, the repeated categorical process follows a \textit{multinomial distribution}. For the $i$-th row of $\textbf{P}_1$, the likelihood function $\mathcal{L}$ is (by omitting the combinatorial factor):
\begin{equation}
\label{eq_likelihood_row_i}
\mathcal{L}(  p_{i,1},\dots,p_{i,m+2} \mid n_{1,1},\dots,n_{1,m+2})=\prod_{j=1}^{m+2} p_{i,j}^{n_{i,j}}
\end{equation}
%\xiaowei{I don't quite understand this likelihood function. }

Upon establishing the likelihood function, many existing estimators can be invoked for our purpose, such as the basic Maximum Likelihood Estimation (MLE) and Bayesian estimators \cite{epifani_model_2009}. While more advanced estimators can be easily integrated in our proposed framework, we only present the use of MLE in this paper for brevity:
\begin{equation}
\label{eq_mle_tranprob}
    \hat{p}_{i,j}=\frac{n_{i,j}}{\sum_{j=1}^{m+2} n_{i,j}}
\end{equation}
It is known that MLE is an unbiased estimator 
%in this case 
\cite{epifani_model_2009}, while the uncertainty in the estimates is captured by the variance that depends on the number of samples. There are also means for calculating $(1-\alpha)$ confidence intervals of the verification results, given the observations on the frequencies between states (exactly as our statistical data $n_{i,j}$). Such result may in turn determine the required number of samples $n_{i,j}$ given a required say 95\% confidence level for the final verification results. Although we did not calculate the confidence interval to determine the sample size in this paper, we instead choose a sample size in our later experiments that is sufficiently large to show a converging trend of the verification results.
%(cf. Section~\ref{sec.5.3}). 

\subsubsection{Construction of Failure Process DTMC}
\label{sec_formalise_DTMC}
%Let $AP=\{term, p_G, p_{B^1},..., p_{B^n}\}$. From a trajectory distribution $\rho^\pi(\zeta)$ induced from a policy $\pi$ and an MDP $\cal{M}$,  we  construct a DTMC $(S,s_G,\textbf{P},L)$ such that $S=\{s_G, s_{B^1},...,s_{B^n}, s_C\}$,  $term \in L(s_C)$, and $p_{B^i}\in L{s_{B^i}}$ for $i\in \{1..n\}$. The probabilistic transition $\textbf{P}$ is defined as $\textbf{P}(s_G,s_1)=$
%\xingyu{double check with Xiaowei...}

The failure process DTMC is the product of two DTMCs, $M_1$ and $M_2$, via the synchronisation of the transition actions.

Let $AP_1=\{crash, neg\_risk, risk\_B_1,\cdots, risk\_B_n\}$, and 
%From a trajectory distribution $\rho^\pi(\zeta)$ induced from a policy $\pi$ and an MDP $\cal{M}$, 
we construct the first DTMC $M_1=(S,s_N,\textbf{P}_1,L_1)$ where
\begin{itemize}
    \item $S=\{s_N, s_{B_1},\dots,s_{B_n}, s_C\}$,
    \item Each entry $p_{i,j}$ of 
%the probabilistic transition matrix 
$\textbf{P}_1$ is defined as Eqn.~\eqref{eq_mle_tranprob}, and 
    \item $neg\_risk \in L_1(s_N)$,  $risk\_B_i\in L_1(s_{B_i})$ for $i\in \{1..n\}$, and
     $crash \in L_1(s_C)$.
\end{itemize}    We also define a reward structure 
%for this DTMC 
%as
{``deviation''}$=(r_S,r_T)$ with
\begin{itemize}
    \item $r_S(s_N)=0$,
    \item $r_S(s_C)=0$,
    \item $r_S(s_{B_i})=d_i$ (where $d_i$ is the deviation from $s_N$ to $s_{B_i}$), and 
    \item $r_T(s1,s2)=0$ for all $s1, s2 \in S$.
\end{itemize} 


%Since the properties 
%of our interest 
%may depend on %in which 
%the stage the \gls{RAS} is during the mission, 
Moreover, we %also 
%formalise 
need a ``mission stage DTMC'' (for simplicity, we only consider two stages---mission terminated or not). Let $AP=\{progressing,terminated\}$, we construct
%a DTMC 
$M_2=(K,k_0,\textbf{P}_2,L_2)$ with
\begin{itemize}
    \item $K=\{k_0,k_1\}$,
    \item $progressing \in L_2(k_0)$ and $terminated \in L_2(k_1)$, and 
    \item The transition probabilities $\textbf{P}_2$ are $p_{k_0,k_1}=\frac{1}{l_{mis}},p_{k_0,k_0}=1-\frac{1}{l_{mis}}, p_{k_1,k_1}=1$ and $p_{k_1,k_0}=0$, where $l_{mis}$ is a constant representing the expected mission length (number of transitions) obtained from the testing data.
\end{itemize}  We also define a reward structure for this DTMC:  {``step''}$=(r_S,r_T)$ with
\begin{itemize}
    \item $r_T(k_0,k_0)=1$,
    \item $r_T(k_0,k_1)=1$, and 
    \item $r_S(k)=0$ for all $k \in K$.
\end{itemize}  

%In later experiments, 
Finally,  
%further 
%encode the two DTMCs in the 
%for model checking, 
we encode the failure process DTMC with PRISM model checker \cite{kwiatkowska_prism_2011}.

\section{Verification of State-Based Policy Robustness}\label{sec:DRLRobustnessverification}

The verification in Section~\ref{sec:DRLverification} mainly concerns whether a learned policy works well in an environment. It actually computes the generalisation ability of the policy in the environment, considering the fact that the policy might not be trained on all paths of the DTMC but is expected to generalise well. Nevertheless, there are other safety properties to be considered. In this section, we consider temporal properties over the state-based robustness. 

For state-based robustness, we concern the the robustness of the policy network (and other supplementary network such as the critic network of DDPG) on individual states. Therefore, temporal properties over the state-based robustness  measure the  dynamics of such robustness of the neural networks when DRL agent interacts with the environment. Actually, state-based robustness is mainly due to the existence of observation noises (from e.g., noisy sensor reading), and we assume that such noise only influences the current state. We will discuss in the next section (Section~\ref{sec:verificationtemporalDRLrobustness}) the case where the noise at a state may influence the future states. 

Formally, to work with state-based policy robustness, we need to lift the model  ${\cal M}^E(\pi,\textbf{x}_0)$ into ${\cal M}^E(\pi,\textbf{x}_0, \mathcal{C})$ to consider the constraint $\mathcal{C}$ on the noise. 

\subsection*{Modelling Noise}




Usually, the constraint $\mathcal{C}(\textbf{x})$ expresses the neighborhood of a state $\textbf{x}$. A neighborhood can be e.g., a distance norm based neighborhood as  \begin{equation}
     \{ \textbf{x}' ~|~ ||\textbf{x} - \textbf{x}'||_p \leq d\},
\end{equation}
a Zonotope expressed as a Minkowski sum of a set of planes, or any other topological shape. Recall that, we use $||\cdot||_p$ to express the $L_p$ norm of a vector for $p\geq 0$. 

An immediate impact of the state-based noise is, instead of having a deterministic action and a single approximated Q value (when given a critic network), there are a set of possible actions and a set of possible approximated Q values. 


\subsection*{Example Properties}

We may consider a few example properties in the following, to quantify the uncertainty incurred by the noise. 

\begin{example}\label{example:robustDRLproperties}
We may write 
property \begin{equation}
    \mathcal{P}_{\leq 0}~\eventually~(\textbf{a}\in \mathcal{A}_{valid})
\end{equation}
to express that it is almost sure that some action $\textbf{a}$ is never activated, where $\mathcal{A}_{valid}$ is the set of actions that are possible at the current state, subject to the noise.  Moreover, we may write 
\begin{equation}
    \mathcal{P}_{\leq 0}~\eventually~(r > c)
\end{equation}
to express that the predictive Q value $r$ (output by the critic network) is always no greater than $c$, subject to the noise at the current state. 
\end{example}

For the remaining of this section, we will discuss how to extend the  DTMC ${\cal M}^E(\pi,\textbf{x}_0)$ into ${\cal M}^E(\pi,\textbf{x}_0,\mathcal{C})$ to work with the properties as in Example~\ref{example:robustDRLproperties}. Actually, we need to expand the set $Prop$ of atomic propositions to include atomic propositions such as $\textbf{a}\in \mathcal{A}_{valid}$ and $r > c$. For this, we need to change the state space from $\mathcal{S}$ to $\mathcal{S}\times \mathcal{P}(\mathcal{S})$, so that we can compute e.g.,  $\mathcal{A}_{valid}$,  $r_{min}$, and $r_{max}$ over expanded states, where $[r_{min},r_{max}]$ is the reachable range of the predictive Q value $r$. 

\subsection*{Expansion of State Space}

Comparing with ${\cal M}^E(\pi,\textbf{x}_0)$, the new model ${\cal M}^E(\pi,\textbf{x}_0, \mathcal{C})$ has an expanded state space $\mathcal{S}\times \mathcal{P}(\mathcal{S})$ such that each state $\textbf{x}$ is now attached with a noise neighborhood $\mathcal{C}(\textbf{x})$. The transition relation depends only on the first element, i.e., on the state $\textbf{x}$, so it can be easily obtained from the transition relation of the model ${\cal M}^E(\pi,\textbf{x}_0)$. Another expansion is on the atomic propositions, which as explained will include some additional atomic propositions as in Example~\ref{example:robustDRLproperties}. 

\subsection*{Evaluation of Atomic Propositions}

Considering a typical DRL agent which has two neural networks $\mu: \mathcal{S}\rightarrow \mathcal{A}$ and $\nu: \mathcal{S}\times \mathcal{A} \rightarrow \mathcal{N}$, representing the actor and the critic agents, respectively. Intuitively, $\mu(\textbf{x})$ returns the action $\textbf{a}$ that needs to be taken on a state $\textbf{x}$, while $\nu(\textbf{x},\textbf{a})$ returns the predictive Q value of taking action $\textbf{a}$ on state $\textbf{x}$. Assume that we have a verification tool $g$ that, given $\mu$ and a constraint $\mathcal{C}(\textbf{x})$ in $\mathcal{S}$, outputs an (over-approximated) reachable set in 
$\mathcal{A}$. That is, we can have 
\begin{equation}\label{equ:reachableactions}
    g(\mu,\mathcal{C}(\textbf{x})) 
\end{equation}
as the reachable set of actions when given a set of states expressed with constraint $\mathcal{C}(\textbf{x})$. With this, we can determine $\textbf{a}\in \mathcal{A}_{valid}$ by knowing whether $\textbf{a}\in g(\mu,\mathcal{C}(\textbf{x})) $.
Such verification tool is available \cite{HUANG2020100270} in previous works \cite{RHK2018,wu2018game,10.1007/978-3-030-32304-2_15}. Moreover, we can compute 
\begin{equation}\label{equ:reachablestates}
    g(\nu,\mathcal{C}(\textbf{x})\times g(\mu,\mathcal{C}(\textbf{x})))
\end{equation}
as the range of predictive Q value, and hence know the values $r_{min}$ and $r_{max}$. Based on them, we can determine if $r>c$.  



%Given a set of observations $\textbf{X}$, we write $\eta(\textbf{X})$ for the minimum shape that contains the set $\textbf{X}$  such that $\textbf{X}\subseteq \eta(\textbf{X})$, where $\eta$ can be e.g., \textbf{Box} and \textbf{Zonotope}, as discussed above.

\subsection*{Probabilistic Model Checking Lifted with New Atomic Propositions}



%In the following, depending on whether learning an environment agent is possible, we have two different ways of computing 

%verification approaches to work with a PCTL safety property \begin{equation}
%    \mathcal{P}_{\geq 1}(true \: U \: p)
%\end{equation} 
%which expresses the almost sure reachability of states satisfying $p$. Note that, $true \: U \: p$ denotes the finite reachability of $p$. 

%\subsection*{Learning Environment Agent}


As explained above, when using neighborhood $\mathcal{C}(\textbf{x})$ to express adding noise to the state $\textbf{x}$, we can lift the DTMC ${\cal M}^E(\pi,\textbf{x}_0)$ into ${\cal M}^E(\pi,\textbf{x}_0, \mathcal{C})$ by associating each state $\textbf{x}$ with a neighborhood $\mathcal{C}(\textbf{x})$. Then, the probabilistic model checking proceeds the same as in Section~\ref{sec:DRLverification} with the  additional consideration of the  neighborhood $\mathcal{C}(\textbf{x})$ on every state as well as the above-mentioned evaluation of atomic propositions. 



\iffalse

To conduct verification, we may have an environment agent $E:\mathcal{S}\times \mathcal{A} \rightarrow \mathcal{P}(\mathcal{S})$, which returns a distribution of next states when given an action on the current state. Such environment can be learned as in \cite{DBLP:journals/corr/abs-1803-10122}. Similar as Equation (\ref{equ:reachableactions}), we have 
\begin{equation}
    g(E, \mathcal{C} \times \mathcal{C}_{\mathcal{A}}) 
\end{equation}
as the reachable set of next states when given a set of states expressed with $\mathcal{C}$ and a set of actions expressed with $\mathcal{C}_{\mathcal{A}}$. 

Based on the above, once we have $E$ and $\mu$, and a set $\mathcal{C}$ of constraints on the current states, we can compute  
\begin{equation}
    \mathcal{C}' = g(E, \mathcal{C} \times g(\mu,\mathcal{C}))
\end{equation}
as the  set of reachable next states. This step can be repeated until obtaining  $\mathcal{C}^{(k)}$ after $k>1$ steps. We can then check whether $(\mathcal{C},\mathcal{C}^{(k)})$ satisfies the constraint $\mathcal{C}$ as defined with e.g., PCTL formulas. 

%\paragraph{Which property to verify?} We consider PCTL property which concerns the expected reward upon $\mu$. 

%\paragraph{Is this a challenging problem?} 

%There are two challenging issues. The first is the lack of environment model. The second is the computation of reachable actions. 

\fi

\section{Verification of Temporal Policy Robustness}\label{sec:verificationtemporalDRLrobustness}

The state-based policy robustness, as discussed in Section~\ref{sec:DRLRobustnessverification}, concerns the dynamics of the robustness of the policy network $\mu$ (and the critic network $\nu$) on individual states of a path.  When the states $\textbf{x}_k$ and $\textbf{x}_{k+1}$ are known, the robustness on state $\textbf{x}_k$ cannot influence the robustness on $\textbf{x}_{k+1}$. This is a simplistic assumption, but can be arguably unrealistic. In this section, we consider a more complex setting about the temporal evolution of robustness, i.e., the robustness on state $\textbf{x}_k$ may influence the robustness of later states $\textbf{x}_{k+1}, \textbf{x}_{k+2}, ...$.  

%When learning an environment agent is infeasible, we can consider an alternative approach. 
To work with temporal evolution of robustness, we cannot re-use the DTMC ${\cal M}^E(\pi,\textbf{x}_0)$. Instead, we need to construct a new Kripke structure \cite{clarkebook} ${\cal M}^E(\pi,\mathcal{C}(\textbf{x}_0))=(S,s_0,T,L)$ from the environment MDP ${\cal M}^E$, the DRL policy $\pi$, the initial state $\textbf{x}_0$, and the noise constraint $\mathcal{C}$. Also, we do not work wih the probabilistic logic PCTL because the transition relation $T$ in $M$ is non-probabilistic, and will work with a non-probabilistic temporal logic LTL. 

\subsection*{State Space}

Given a set of MDP states $\textbf{X}$, we write $\eta(\textbf{X})$ for the minimum polytope that contains the set $\textbf{X}$, i.e., $\textbf{X}\subseteq \eta(\textbf{X})$, where $\eta$ can be e.g., \textbf{Box} and \textbf{Zonotope}, as discussed above. Intuitively, $\eta(\textbf{X})$ is an over-approximation of $\textbf{X}$. Let $S$ of ${\cal M}^E(\pi,\mathcal{C}(\textbf{x}_0))$ be the set of states such that every state $s$ is a polytope $\eta(\textbf{X})$.  The function $\eta$ is used to keep all the states in ${\cal M}^E(\pi,\mathcal{C}(\textbf{x}_0))$ as the same polytope shape, to make the state space manageable. With this definition of states $S$, the initial state $
s_0= \eta(\mathcal{C}(\textbf{x}_0))$ is the polytope containing the noise neighborhood of the initial observation $\textbf{x}_0$. 

\subsection*{Transition Relation}

First of all, we need two supplementary functions. 
The first function $h_1$ is, given a constraint $\mathcal{C}$, to split $\mathcal{C}$ into a set of constraints, i.e.,  
\begin{equation}
    h_1(\mathcal{C},g,\mu) = \{\mathcal{C}^{\textbf{a}_1}, ..., \mathcal{C}^{\textbf{a}_m}\}
\end{equation}
such that $\mathcal{C} = \mathcal{C}^{\textbf{a}_1}\cup ... \cup \mathcal{C}^{\textbf{a}_m}$ and, for all $1\leq i\leq m$, $g(\mu,\mathcal{C}^{\textbf{a}_i})=\{\textbf{a}_i\}$. Intuitively, after splitting, all states in a constraint $\mathcal{C}^{\textbf{a}_i}$ take the same action $\textbf{a}_i$. We write $a(\mathcal{C}^{\textbf{a}_i})=\textbf{a}_i$. The second function $h_2$ is, given constraints $\mathcal{C}$ and an action $\textbf{a}$, to return the set of next states, i.e.,  
\begin{equation}
    h_2(\mathcal{C},\textbf{a}) =  \bigcup_{\textbf{x}\in \mathcal{C}} support(\mathcal{P}(\textbf{x},\textbf{a})) 
\end{equation}
where $\textbf{x}\in \mathcal{C}$ means that the state $\textbf{x}$ satisfies the constraint $\mathcal{C}$, and $support(D)$ returns the support of a distribution $D$. We recall that $\mathcal{P}(\textbf{x},\textbf{a})$ is the probabilistic transition function of ${\cal M}^E$. We remark that, $h_1$ can be obtained by applying binary search together with the verification tool $g$, and $h_2$ can be obtained by a genetic algorithm to search for a convex hull over the next states. 

Based on $h_1$ and $h_2$, we can construct the transition relation $T$ of ${\cal M}^E(\pi,\mathcal{C}(\textbf{x}_0))$. For every suitable polytope $\mathcal{C}$, we let 
\begin{equation}
    (\mathcal{C},\mathcal{C}')\in T\text{ for every }\mathcal{C}'\in \{\eta(h_2(\mathcal{C}^{\textbf{a}},\textbf{a}))~|~\mathcal{C}^{\textbf{a}} \in h_1(\mathcal{C},g,\mu)\}.
\end{equation} Note that, with this construction, we assume that the noise is only from the initial state, without having any additional noise in the later steps. If we need to consider additional noise for every step, we may replace the set $\{\eta(h_2(\mathcal{C}^{\textbf{a}},\textbf{a}))~|~\mathcal{C}^{\textbf{a}} \in h_1(\mathcal{C},g,\mu)\}$ with $\{\eta(\mathcal{C}(h_2(\mathcal{C}^{\textbf{a}},\textbf{a})))~|~\mathcal{C}^{\textbf{a}} \in h_1(\mathcal{C},g,\mu)\}$, i.e., every input subset $h_2(\mathcal{C}^{\textbf{a}},\textbf{a})$ is added with noise represented with the constraint $\mathcal{C}$.


\subsection*{LTL Model Checking}

The labelling function $L$ of ${\cal M}^E(\pi,\mathcal{C}(\textbf{x}_0))$ is the same as the one in the DTMC ${\cal M}^E(\pi,\textbf{x}_0)$ for atomic propositions such as $\textbf{a}\in \mathcal{A}_{valid}$ and $r > c$. The Kripke structure ${\cal M}^E(\pi,\mathcal{C}(\textbf{x}_0))$ is non-probabilistic. Therefore, we can apply LTL model checking~\cite{clarkebook} to understand the qualitative properties such as 
\begin{equation}
    \neg \eventually\neg ~(\textbf{a}\notin \mathcal{A}_{valid})
\end{equation}
which expresses that action $\textbf{a}$ will never occur, and 
\begin{equation}
    \neg \eventually\neg ~(r \leq  c)
\end{equation}
which expresses that the predictive Q value $r$ is always no greater than $c$. 

\iffalse

Now, given a constraint $\mathcal{C}$ on the current states, we determine the safety of   $\mathcal{C}$, written as $\mathcal{C}\models \mathcal{C}$, with 
\begin{equation}\label{equ:spliting}
    \bigwedge_{\mathcal{C}\in h_1(\mathcal{C},g,\mu)} h_2(\mathcal{C},a(\mathcal{C}))\models \mathcal{C}
\end{equation}
Note that, Equation (\ref{equ:spliting}) is a recursive definition, i.e., $h_2(\mathcal{C},a(\mathcal{C}))\models \mathcal{C}$ can also be computed in this way. By recursively computing Equation (\ref{equ:spliting}) for $k$ steps, we can have the verification result. 



\fi



\section{Addressing Sim-to-Real Challenge}\label{sec:verificationsim2real}

Considering the lack of real-world training data for DRL agents, it is a common practice to train a DRL policy in a simulation environment and then consider its application in real-world settings by transferring the knowledge and adapting the policy. More and more high-fidelity simulation platforms such as  AirSim \cite{10.1007/978-3-319-67361-5_40}, CARLA~\cite{DBLP:journals/corr/abs-1711-03938}, and RotorS~\cite{Furrer2016} have been made available, in order to reduce the gap between simulation and reality.  Nevertheless, the gap exists and technical means to detect and reduce the gap are still needed. Typical methods to reduce the gap during training include e.g., adding perturbances to the environment \cite{DBLP:journals/corr/abs-2008-07875,ZHAO2020324,DBLP:journals/corr/abs-1810-01032}, building a precise mathematical model for a physical system \cite{179842,9341260}, domain randomisation \cite{DBLP:journals/corr/abs-2003-02471}, domain adaptation \cite{9196540}, policy distillation \cite{DBLP:journals/corr/abs-1906-04452}, and consideration of novel experiences that were not in the simulations \cite{DBLP:journals/jair/RamakrishnanKDH20}. 
%Moreover, meta learning~\cite{9196540} and continual learning~\cite{DBLP:journals/corr/abs-1906-04452} have also been applied to deal with Sim2Real challenge. 

Unlike the methods to reduce the gap, less works have been done on detection, i.e., identifying the gap and evaluating the extent of the gap. The verification method in Section \ref{sec:DRLverification} can be used to quantitatively evaluate how well a DRL policy $\pi$ executes in an environment $E$. Because $E$ can be the real world or the simulation environment,  we can construct two models ${\cal M}^{E_{simu}}(\pi,\textbf{x}_0)$ and ${\cal M}^{E_{real}}(\pi,\textbf{x}_0)$, and use their difference such as \begin{equation}
    \mathcal{P}_{=?}^{E_{simu}}(\psi) - \mathcal{P}_{=?}^{E_{real}}(\psi), 
\end{equation} to measure the gap between two environments $E_{simu}$ and $E_{real}$ over the probability of satisfying the property $\psi$. With this, the gap is exhibited roughly in the model view ($E_{simu}$ vs $E_{real}$). 

On the other hand, the verification methods in Section~\ref{sec:DRLRobustnessverification} and Section~\ref{sec:verificationtemporalDRLrobustness} concern the extent to which the environment noise, which may not appear in the simulation environment but will appear in the real world, may affect the properties. The sim-to-real gap can be  exhibited by checking the difference between the probabilities of the properties  before and after the consideration of environment noise. Similarly, this is also on model view ($E_{simu}$ vs $E_{simu+noise}$). 


However, none of the above methods is able to directly work with paths, to identify specific paths on which a path property $\psi$ has different satisfiability results in different environments, or to compute the probability of such paths. 
We require that on such paths, the DRL agents take the same sequence of actions in different environment. Identifying these paths are essential, because they are direct evidence of the existence of gap between environments. 

\subsection*{Model Construction}

To enable our working with the above-mentioned paths, we can construct another DTMC ${\cal M}^{E_1\times E_2}(\pi,(\textbf{x}_0,\textbf{x}_0))=(S_1\times S_2,(\textbf{x}_0,\textbf{x}_0),\textbf{P},L)$, which is a synchronisation of the two DTMCs ${\cal M}^{E_1}(\pi,\textbf{x}_0)=(S_1,\textbf{x}_0,\textbf{P}_1,L_1)$  and ${\cal M}^{E_2}(\pi,\textbf{x}_0)=(S_2,\textbf{x}_0,\textbf{P}_2,L_2)$ such that 
\begin{itemize}
    \item the transition relation is ${P}((\textbf{x}_1,\textbf{x}_2),(\textbf{x}_1',\textbf{x}_2')) = {P}_1(\textbf{x}_1,\textbf{x}_1')*{P}_2(\textbf{x}_2,\textbf{x}_2')$, with the requirement that $\mathcal{P}_1(\textbf{x}_1'~|~\textbf{x}_1,\textbf{a}) > 0 $ and $\mathcal{P}_2(\textbf{x}_2'~|~\textbf{x}_2,\textbf{a}) > 0 $ for some action $\textbf{a}$. $\mathcal{P}_1$ and $\mathcal{P}_2$ are the probability transition relations of the MDPs ${\cal M}^{E_1}$ and ${\cal M}^{E_2}$, respectively. We note that, a normalisation may be needed to  retain a probability distribution for every state $(\textbf{x}_1,\textbf{x}_2)$. 
    \item the labelling function $L$ will return a Boolean value for every atomic proposition on every state $(\textbf{x}_1,\textbf{x}_2)$, indicating whether or not the atomic proposition is evaluated  the same on the two states $\textbf{x}_1$ and $\textbf{x}_2$, respectively, over their DTMCs. 
\end{itemize}

\subsection*{Properties}

Based on the construction of ${\cal M}^{E_1\times E_2}(\pi,(\textbf{x}_0,\textbf{x}_0))$, we may verify some temporal properties such as 
\begin{equation}
    \neg \eventually \neg (r>c)
\end{equation}
which expresses that the robot always has the same evaluation on $r>c$ in two environments. Any counterexample to this will lead to a finite path where $r>c$ maintains the same evaluation until reaching a state $(\textbf{x}_1,\textbf{x}_2)$ where it is evaluated differently on $\textbf{x}_1$ and $\textbf{x}_2$, respectively. Probabilistic properties may also be considered, such as 
\begin{equation}
    \mathcal{P}_{=?}~\neg \eventually\neg ~(r > c)
\end{equation}
which returns the probability of maintaining $r>c$  across the environments. 



\chapter{Testing Techniques}\label{chap:testing}

Verification techniques, as discussed in Chapters~\ref{part:deep} and \ref{chap:drl}, are to ascertain -- with mathematical proof --  whether a property holds on a mathematical model. The soundness and completeness required by the mathematical proof result in the scalability problem that verification algorithms can only work with either small models (e.g., the MILP-based method as in Chapter~\ref{chap:MILP}) or limited number of input dimensions (e.g., the reachability analysis as in Chapter~\ref{chap:reachabilityAnalysis}). In practice, when working with real-world systems where the machine learning models are large in nature, other techniques have to be considered for the certification purpose. 
%
Similar to traditional software testing against software verification, neural network testing provides a certification methodology with a balance between completeness  and efficiency. In established industries, e.g., avionics and automotive, the needs for software testing have been settled in various standards such as DO-178C and MISRA. However, due to the lack of logical structures and system specification, it is less straightforward on how to extend such standards to work with systems with neural network components. In the following, we discuss some existing neural network testing techniques. The readers are referred to the survey~\cite{HUANG2020100270} for more discussion. 



\section{A General Testing Framework}\label{sec:testframework}

Assume that according to the model $f$ and a safety property $\phi$, we are able to define a set of test objectives $\requirements$. We will discuss later in Section~\ref{sec:test-criteria} a few existing covering methods $\coveringmethod$ to define $\requirements$ for convolutional neural networks. 

\begin{definition}[Test Suite]
Given a neural network $f$, a test suite $\testsuites$ is a finite set of input instances, i.e., $\testsuites\subseteq \mathcal{D}$. Each instance is called a test case. 
\end{definition}

%Usually, a test case is a single input $\testsuites\subseteq D_{1}$, e.g., in \cite{PCYJ2017}, or a pair of inputs $\testsuites\subseteq D_{1}\times D_{1}$, e.g., in \cite{sun2018testing-b}. 
Ideally, given the set of test objectives
$\requirements$ with respect to some covering method $\coveringmethod$, %$\requirements=(\coveringmethod,\testobjectives)$ 
we run a test case generation algorithm (to be introduced in Section~\ref{sec:testcasegen}) to find a test suite $\testsuites$ such that %
  \begin{equation}\label{equ:testframework}
  \forall \neuronpair\in \requirements \exists \textbf{x} \in \testsuites: \coveringmethod(\neuronpair,\textbf{x})
 \end{equation} 
where $\coveringmethod(\neuronpair,\textbf{x})$ intuitively means that the test objective $\neuronpair$ is satisfied under the test case $\textbf{x}$. Intuitively, Equation (\ref{equ:testframework}) means that every test objective is covered by some of the test cases. 
In practice, we might want to
compute the degree to which the test objectives are satisfied by a test suite $\testsuites$.


\begin{definition}[Test Criterion]
Given a neural network $f$, a covering method $\coveringmethod$, a set $\requirements$ of test objectives, % = (\coveringmethod,\testobjectives)$,
and a test suite $\testsuites$,  the test criterion $\metric_{\coveringmethod}(\requirements,\testsuites)$ is as follows: 
\begin{equation}
  \label{eq:madc}
  \metric_{\coveringmethod}(\requirements,\testsuites)=\frac{|\{\neuronpair \in \requirements | \exists  \textbf{x} \in \testsuites: \coveringmethod(\neuronpair,\textbf{x})\}|}{|\requirements|}
\end{equation}
\end{definition}
 
Intuitively, it computes the percentage of the test objectives that are
covered by test cases in $\testsuites$ w.r.t. the covering method~$\coveringmethod$. 


\section{Coverage Metrics for Neural Networks}
\label{sec:test-criteria}

Research in software engineering has resulted in a broad range of approaches
to testing software. Please refer to \cite{ZHM1997,JH2011,SWMPHS2017}  for comprehensive reviews.
In white-box testing, the structure of a
program is exploited to (perhaps automatically) generate test cases.
Structural coverage criteria (or metrics) define a set of test objectives to be covered, guiding the generation of test cases and evaluating the completeness of a test suite. 
E.g., a test suite with 100\% statement coverage exercises all statements of the program at
least once.  While it is arguable whether this ensures functional correctness,
high coverage is able to increase users' confidence (or trust) in the testing results~\cite{ZHM1997}.  Structural coverage analysis and testing are also
used as a means of assessment in a number of safety-critical scenarios, and
criteria such as statement and modified condition/decision coverage (MC/DC) are
applicable measures with respect to different criticality levels.  MC/DC was developed by NASA\cite{HVCR2001} and has been widely adopted.  It is used in avionics software development guidance to ensure adequate testing of applications with the highest criticality \cite{do178}.

%We let $\setofcoveringmethods$ be a set of covering methods, and
Let $\requirements$ be the set of test objectives to be covered. For different structure coverage, we can define different sets of test objectives. In \cite{PCYJ2017}, $\requirements$ is instantiated as the set of statuses of hidden neurons. That is, for the set $\mathcal{H}_i$ of hidden neurons with ReLU activation functions at layer $i$, we let 
\begin{equation}
    \requirements_{\text{neuron coverage}}=\bigcup_{i=2}^{K-1}\mathcal{H}_i
\end{equation}
Intuitively, combining with the general testing framework in Section~\ref{sec:testframework}, it requires to find a test suite $\testsuites$ such that for every hidden ReLU neuron, there is a test case $t\in \testsuites$ who can activate it.  
As another example, in \cite{sun2018testing-b},  $\requirements$ is instantiated as the set of causal relationships between feature pairs. That is, 
\begin{equation}
    \requirements_{\text{MC/DC coverage}}=\bigcup_{i=2}^{K-2}\{(h_1,h_2)~|~h_1\in \mathcal{H}_i, h_2\in \mathcal{H}_{i+1}\}
\end{equation}
Intuitively, combining with the general testing framework in Section~\ref{sec:testframework}, it requires to find a number of test pairs  such that  each $(h_1,h_2)\in \requirements_{\text{MC/DC coverage}}$, 
$h_2$ can be independently activated by $h_1$  \cite{DBLP:journals/corr/abs-1803-04792}.

%\begin{definition}[Test  Conditions] \label{def:coverageRequirement}
%Given a neural network $f$ and a  covering method $\coveringmethod\in \setofcoveringmethods$, a test objective set $\requirements$ is characterised by a pair $(\coveringmethod,\testobjectives)$ such that $\testobjectives\subseteq \neuronpairs(f)$.
%\end{definition}
 
%Intuitively, a test objective  $(\coveringmethod,\testobjectives)$ is to ask for the coverage of test objectives in $\testobjectives$ with the covering method $\coveringmethod$.  

\section{Test Case Generation}\label{sec:testcasegen}

Once a coverage metric is determined, it is needed to develop a test case generation algorithm to produce a set of test cases. Existing methods include e.g., input mutation~\cite{wicker2018feature}, fuzzing~\cite{odena2018tensorfuzz}, genetic algorithm~\cite{9451178}, symbolic execution~\cite{sun2018concolic}, and gradient ascent~\cite{sun2018concolicb}. 



\section{Discussion}

In addition to testing techniques which originate from the software engineering area, there are other techniques that might be useful to analyse the safety of neural networks. 
%
Statistical evaluation applies statistical methods in order to gain insights into the verification problem we concern. In addition to the purpose of determining the existence of failures in the deep learning model, statistical evaluation assesses the satisfiability of a property in a probabilistic way, by e.g.,
%obtaining assessment result via
aggregating sampling results. The aggregated evaluation result may have probabilistic guarantee, in the form of e.g., the probability of failure rate lower than a threshold $l$ is greater than $1-\epsilon$, for some small constant $\epsilon$. 



For the robustness, sampling methods, such as \cite{weng2018evaluating},
%,webb_statistical_2019}, 
are to summarise property-related statistics from the samples. 
%
While sampling methods can have probabilistic guarantees via e.g., Chebyshev's inequality, it is still under investigation on how to associate test coverage metrics with probabilistic guarantee. 
%
For the generalisation error, other than the empirical approach of using a set of test data to evaluate, recent efforts on complexity measure \cite{chatterji2019intriguing,jin2020does} suggest that it is possible to estimate generalisation error -- with theoretical bound -- by only considering the weights of the deep learning without resorting to the test dataset. 



\chapter{Reliability Assessment}\label{sec:safetyassurance}


In Chapters~\ref{part:deep}, \ref{chap:drl}, and  \ref{chap:testing}, we have introduced  verification and testing techniques for convolutional neural network (CNN) and deep reinforcement learning (DRL).  These techniques are to work with individual safety properties, such as the robustness of a data instance $\textbf{x}$ (for CNN) or the safety of a policy $\pi$ on a given initial state $\textbf{x}_0$ (for DRL). However, considering that deep learning models usually serve as components of a large autonomous system, the safety of the deep learning models is related to how it is used in the system. For example, if a CNN is used as a perception component for e.g., object detection in a self-driving car, there will be a set $D_{op}$ of data instances that may appear in operational time, all of which needs to be verified. If a DRL agent is used as a control component for e.g., navigation in a mobile robot, there will be a set of possible trajectories that may appear in operational time, all of which needs to be verified.  
%it is possible to operate in an environment that is different from the environment where it is trained. 
%
In this chapter, we introduce a principled approach to utilise evidence produced by verification techniques about low-level safety properties (e.g., robustness of individual instances) to reason about high-level safety claims, such as ``the perception component of the self-driving car can correctly classify the next 1,000 instances with probability higher than 99\%''. These high-level safety claims are required in various industrial standards for software used in safety-critical systems. 
%
Methodologically, the approach is based on the safety argument, which provides a link between the safety evidence and a safety claim, showing that the safety evidence is sufficient to support the claim. 


\section{Reliability Assessment for Convolutional Neural Networks}\label{sec:assuranceCNN}

Assume that, we are working with a verification technique $g$, which, given a network $f$ and a constraint $\mathcal{C}$, returns the probability of the inputs within $\mathcal{C}$ being classified correctly. The constraint $\mathcal{C}$ can be e.g., a norm ball with $\textbf{x}$ as the centre to denote the possible perturbations. 


Also, as mentioned above, we assume that there are a set $D_{op}$ of operational data instances. 
%
We partition the input domain $\mathcal{D}$ into $m$ cells, subject to the $r$-separation property \cite{pietrantuono_reliability_2020}. These cells are disjoint and altogether form the entire input domain. Let $p_{op}$ be the empirical distribution of the cells estimated with the dataset $D_{op}$. Then, we can learn a generative model $G_{\theta}$ over parameters $\theta$ such that 
\begin{equation}
    \theta^* = \argmin_{\theta}  \text{KL}(G_{\theta}, p_{op})
\end{equation}
where $\text{KL}(\cdot,\cdot)$ is the KL divergence between two distributions. 

Based on the above, we can estimate the reliability (defined as the probability of failure in classifying the next input) as 
\begin{equation}
   \text{Reliability}(f) = \sum_{i=1}^m G_{\theta}(\mathcal{C}_i)(1-g(f,\mathcal{C}_i))
\end{equation}
Intuitively, $G_{\theta}(\mathcal{C}_i)$ returns the probability density of the cell $i$ represented as the constraint $\mathcal{C}_i$, and $1-g(f,\mathcal{C}_i)$ returns the failure rate of the neural network $f$ working on inputs satisfying the constraint $\mathcal{C}_i$. 

\cite{zhao_safety_2020,DBLP:journals/corr/abs-2112-00646} show how to develop a principled safety argument %method for  in order 
to justify the reliability claim by aggregating evidence from either formal verification or statistical evaluation with Bayesian inference \cite{strigini_software_2013}. Other than learning a generative model, there are other methods to learn the  distribution of cells~\cite{zhao_safety_2021}.


\section{Reliability Assessment for Deep Reinforcement Learning}\label{sec:DRLrealibility}

The execution of a DRL-driven robot in an environment leads to a trajectory distribution (modelled as a distribute-time Markov chain, as discussed in Chapter~\ref{chap:drl}), where the uncertainty (modelled with probability distribution) is from the environment\footnote{For simplicity, we assume DRL policy is deterministic. There are DRL policies which are probabilistic, but the deterministic assumption is without loss of generality, and the methods we develop in this chapter can be easily adapted to work with probabilistic policies.}. Formally, given an environment $E$, a policy $\pi$, and an initial state $\textbf{x}_0$, we can construct a model ${\cal M}^E(\pi,\textbf{x}_0)$ representing the probability distribution of a set of trajectories. Assume that we have a verification technique $g$, as discussed in Sections~\ref{sec:DRLverification}, \ref{sec:DRLRobustnessverification}, and \ref{sec:verificationtemporalDRLrobustness}. 


\begin{definition}
The verification problem is, given a constructed model ${\cal M}^E(\pi,\textbf{x}_0)$ and a verification tool $g$, to determine whether the model is safe with respect to certain property $\phi$, written as ${\cal M}^E(\pi,\textbf{x}_0)\models^{g} \phi$. We may omit the superscript $g$ and write ${\cal M}^E(\pi,\textbf{x}_0)\models \phi$, if it is clear from the context.  We can also assume that $g$ returns a probability value -- a Boolean answer can be converted into a Dirac probability. Then, the verification problem is to compute $Pr({\cal M}^E(\pi,\textbf{x}_0), \phi)$, i.e., the probability of safety. 
\end{definition}



In the following, we discuss how the above verification problem may contribute to the computation of the reliability.  Similar as Section~\ref{sec:assuranceCNN}, we partition the set of possible initial states into $m$ subsets, each of which is represented as a constraint $\mathcal{C}_i$, for $i=1..m$. Then, we can also define the empirical distribution $p_{op}$ over the partitions, and find a model $G_{\theta}$ that is as close as possible to $p_{op}$. Formally, assume that $G_{\theta}$ is a generative model over parameters $\theta$, we have  
\begin{equation}
    \theta^* = \argmin_{\theta}  \text{KL}(G_{\theta}, p_{op})
\end{equation}
where $\text{KL}(\cdot,\cdot)$ is the KL divergence between two distributions. Let $\textbf{x}_{\mathcal{C}_i}$ be the central point (i.e., a representative) of $\mathcal{C}_i$. 

Based on these, we can estimate the reliability (defined as the probability of failure in satisfying $\phi$ with the policy $\pi$ in the environment $E$) as 
\begin{equation}
   \text{Reliability}(E,\pi,\phi) = \sum_{i=1}^m G_{\theta}(\mathcal{C}_i)(1-Pr({\cal M}^E(\pi,\textbf{x}_{\mathcal{C}_i}), \phi))
\end{equation}
where $G_{\theta}(\mathcal{C}_i)$ returns the probability density of the partition $i$ that is  represented as the constraint $\mathcal{C}_i$, and $1-Pr({\cal M}^E(\pi,\textbf{x}_{\mathcal{C}_i}), \phi)$ returns the failure rate of the DRL agent $\pi$ working on inputs satisfying the constraint $\mathcal{C}_i$ under the environment $E$. 

Note that, $G_{\theta}$ can be estimated in the same way as the data distribution in the convolutional neural networks. The discussion on the computation of ${\cal M}^E(\pi,\textbf{x}_0)\models \phi$ or $Pr({\cal M}^E(\pi,\textbf{x}_0), \phi)$ is in Chapter~\ref{chap:drl}.



\chapter{Assurance of Machine Learning  Lifecycle}\label{chap:safetyassurance}

Up to now, all the techniques discussed are working with subjects at certain stage of the machine learning development cycle. For example, verification techniques are working with trained models, and adversarial training is working with models when learning. However, to ensure the safety of critical systems, safety assurance is usually required to assure the  development lifecycle and demonstrate to others (such as third party clients and authorities) that the  system performs accordingly. 
In general, safety assurance activities include  systematic processes for continuous monitoring and recording of the system's safety performance, as well as evaluation of the safety processes and practices. In terms of the machine learning systems, safety assurance activities are to monitor, evaluate, and enforce safety measures for their lifecycle. 
%
%This chapter will cover other safety related topics that have not been discussed before, and mainly focus on the safety assurance on the machine learning lifecycle.  
Figure~\ref{fig:lifecyle} presents the four stages in machine learning cycle when considering their working in safety critical systems: data preparation, model construction and model training, verification and validation, and runtime enforcement.
In this chapter, we will discuss some perspectives (e.g., good practice, safety measurement) for each lifecycle stage, as indicated in Figure~\ref{fig:lifecyle}. For example, for data preparation stage, we will present a workflow of good practice in preparing the training dataset, and discuss a measurement on determine its quality (i.e., the sufficiency). For runtime enforcement, we will discuss a few aspects that are essential to the safe deployment of the machine learning model in an environment, focusing on the partnership of AI and humans. We also discuss expected outcome of each stage. 


\begin{figure}[!htbp]
    \centering
    \includegraphics[width=0.8\textwidth]{images/LookFurther/lifecycle.png}
    \caption{Safety Assurance of Machine Learning Lifecycle}
    \label{fig:lifecyle}
\end{figure}



\section{Assurance on Data}\label{sec:datapreparation}

Most efforts on the analysis of machine learning models, as we discussed in previous chapters, are on the trained models. This is based on an assumption that, the trained models have taken into consideration all necessary information from the training dataset; no more, no less. Any deviation from this assumption may lead to the analysis results not applicable to the machine learning development cycle. For example, if the generated test cases are not on the same distribution with the training data, then the reliability assessment results (Section~\ref{sec:safetyassurance}) such as the failure rate based on the test cases will not be valid. If the training dataset does not conform with the operational data distribution, then the empirical generalisation error will not be a good approximation to the true generalisation error (Section~\ref{sec:generalisationerror}). There are also some safety vulnerabilities such as data poisoning that cannot be detected simply from a trained model. While an exact computation, or even an  estimation, on the deviation is hard to achieve, we believe a few  data processing steps are useful as \emph{good practice} to improve the quality of training data and are therefore essential for safety assurance of machine learning. These steps include data cleaning, data pre-processing, data augmentation, and data anomaly detection. Figure~\ref{fig:dataquality} provides an illustrative diagram showing the flow of data in these steps. 

\begin{figure}[!htbp]
    \centering
    \includegraphics[width=\textwidth]{images/LookFurther/dataquality.png}
    \caption{Assurance on Data}
    \label{fig:dataquality}
\end{figure}

%
\emph{Data cleaning} is a process of ensuring that data is correct, consistent and usable. There are some recent good practices in industry. For example, \cite{10.14778/3229863.3229867} discusses a good practice in Amazon on automating the data validation process, where the quality of data is measured from three aspects: completeness, consistency, and accuracy. The completeness refers to the degree to which a data instance includes data required to describe a real world object, the consistency is defined as the degree to which a set of semantics rules are violated, and the accuracy is the correctness of the data. A set of pre-defined constraints are utilised by the user to define more involved data quality constraints, over which one can determine the quality of a dataset. 

After passing the data cleaning, the dataset is \emph{pre-processed} with  a few typical methods, such as standarisation, normalisation,  whitening, and decorrelation. These techniques are to optimise the dataset in order to help machine learning algorithms to achieve better performance. 
%

The data cleaning and pre-processing steps will result in a dataset that can be utilised for training. However, we still need to ensure that, for a  target machine learning algorithm, the data is sufficient for training a good model. Different machine learning algorithms may have different data sufficiency requirements. A principled decision process (to be discussed further in Section~\ref{sec:trainingsufficiency}) is required to determine the sufficiency of training data. If it is believed that the training data is insufficient, additional efforts may be needed on data collection. On the other hand, for some safety properties such as the robustness, additional data may come from  \emph{data augmentation}, which generates more data during the training phase so that the resulting trained model performs better on the safety property. However, unlike safety properties, extra care is needed to use data augmentation for the improvement of the accuracy of the model, because it will risk losing the i.i.d. property of training data. 

Finally, related to the safety of machine learning, we need to be mindful on the potential of data poisoning and backdoor attacks, and it is recommended that some techniques (as we discussed in the previous chapters) for the detection and reduction of such risks would be desirable.  Along this line, \cite{DBLP:conf/mlsys/BreckP0WZ19} considers a good practice at Google on the quality of data that will be fed into machine learning pipeline. It mainly focuses on utilising an anomaly detector to deal with challenges such as unexpected patterns, schema-free data, and training/serving skew. 

\subsection*{Expected Outcome}

The resulting dataset is required to be complete, consistent, accurate, optimised, sufficient, and free from outliers and poisoning data. All these properties can, and should, be objectively evaluated. We remark that, there are other requirements, such as balanced data and data diversity, which are also desirable to have 
%but can be arguable to be necessary 
for the safety assurance. If additional requirements are to be imposed, objective measurements and validation methods are needed. 

\section{Sufficiency of Training Data}\label{sec:trainingsufficiency}

Given a target machine learning algorithm and a set of training data, it is imperative to understand if the training dataset is sufficient for the training of a good machine learning model. While there are some rules of thumb that may be applicable, for example, the training data needs to be 10 times over the number of trainable parameters, it is believed that more principled methods are needed. We remark that, when the training data is insufficient for a given machine learning algorithm, it is easy to make the resulting trained model overfitted, although there are some recent discussions on the observation that over-parameterised neural network does not overfit \cite{NEURIPS2019_62dad6e2}. 

\subsection*{VC-dimension}

First of all, theoretical results such as Vapnik–Chervonenkis dimension (or VC-dimension) are available to roughly estimate the required number of training data. Specifically, VC-dimension measures the capacity (or expressive power) of a machine learning model. It is defined as the cardinality of the largest set of data points that the machine learning algorithm can shatter. Formally, a machine learning model $f$ parameterised with $\theta$ is said to shatter a set of data instances if for all assignments of the labels to the data instances, there exists a valuation of $\theta$ such that $f$ makes no error when classifying the set of data instances. The readers are referred to Section~\ref{sec:perceptronexpressivity} for an intuitive explanation on the expressivity of a perceptron. For a perceptron of $k$ variables (for example, $k=2$ as in Table~\ref{tab:truthtableand}), the VC-dimension is $k+1$.  

Once we compute the VC dimension $d_{VC}$ for a model $f$, we have the following VC-bound \cite{Vapnik2015}: 

\begin{equation}\label{equ:VCdimension}
    \displaystyle Pr\left( Err(f,\mathcal{D}) \leq Err(f,D_{train}) + \sqrt{\frac{1}{N}[ d_{VC}(\log(\frac{2N}{d_{VC}})+1)-\log(\frac{\epsilon}{4})}]\right ) = 1 - \epsilon 
\end{equation}
where $1-\epsilon$ is an arbitrary probability expressing the confidence, and $N$ is the number of data instances. Then, according to Equation (\ref{equ:VCdimension}), given the required error $Err(f,\mathcal{D})$ and confidence $1-\epsilon$ from the problem, we are able to estimate the number $N$ of training data instances. 

We remark that, in addition to VC-dimensions, there are other machine learning theoretical methods, such as Rademacher complexity and PAC Bayes theory, that can be utilised for this purpose. 

\subsection*{Learning Curve}

In addition to the estimation through VC-dimension, there are empirical ways to determine the size of training dataset. For example, learning curves (as we explained in Figure~\ref{fig:learning_curve} as an example) presents the model performance as the function of the training dataset size. By plotting the learning curve, we are able to 
%determine the size of training dataset by monitoring 
monitor the convergence of the curve. Once the learning curve is converged (with e.g., an $\epsilon$ termination threshold), the increase of training dataset size will not lead to significantly improved performance, and it is regarded as the sufficiency of the training data. 





\section{Optimised Model Construction and  Training}\label{sec:archNtraining}

In this section, we focus on the feedforward neural networks. Once the training dataset is ready, the developer needs to construct a neural network, decide on the training hyper-parameters, and then apply optimisation algorithm to train the neural network. In the following, we start with an analytical analysis of them through a decomposition of the generalisation error (Section~\ref{sec:generalisationerror}). 

We write $G^{0-1}_{f}=GE(f,D_{train},D_{test})$ for the 0-1 generalisation error, and $\networks$ for the set of possible neural networks. Then, $G^{0-1}_{f}$ can be decomposed as follows: 

\begin{equation}
\label{eq_decomp_ge}
G^{0-1}_f =
\underbrace{G^{0-1}_{f} -\inf_{f \in \networks}G^{0-1}_f}_\text{Estimation error of $f$}
+
\underbrace{\inf_{f \in \networks}G^{0-1}_f-G^{0-1,*}_{D_{train}}}_\text{Approximation error of $\networks$}
+\underbrace{G^{0-1,*}_{D_{train}}}_\text{Bayes error}
\end{equation}
where $G^{0-1,*}_{D_{train}}$ is the 0-1 generalisation error of the Optimal Bayes classifier. The \textit{Bayes error} is the lowest and irreducible error over all possible classifiers for the given classification problem \cite{fukunaga_introduction_2013}. It is non-zero if the true labels are not deterministic (e.g., an image being labelled as $y_1$ by one person but as $y_2$ by others), thus intuitively it captures the uncertainties in the dataset $D_{train}$ and the true distribution $\mathcal{D}$ when aiming to solve a real-world problem with machine learning. 
%We remark that, The assurance of data in Section~\ref{sec:datapreparation} can help alleviate this error, but cannot completely reduce it. 
%
%We estimate 
%this error
%(implicitly) 
%it 
%at the \textbf{initiation} and \textbf{data collection} stages in activities like: necessity consideration and dataset preparation etc.
%
%
The \textit{approximation error of $\networks$} measures how far the best classifier in $\networks$ is from the overall optimal classifier, after isolating the Bayes error. The set $\networks$ is determined by the architecture of the machine learning model, thus lifecycle activities at the \textbf{model construction} stage are used to minimise this error.
%, essentially.
Finally, the \textit{estimation error of $f$} measures how far the learned classifier $f$
is from the best classifier in $\networks$. Lifecycle activities at the \textbf{model training} stage 
%after the DNN being constructed 
essentially aim to reduce this error.
%, i.e., 
%performing
%doing
%optimisations 
%in 
%of 
%the set $\networks$.

Both the approximation and estimation errors are reducible. The \emph{ultimate goal} of all lifecycle activities is to reduce the two errors to 0, especially for safety-critical systems. This is analogous to the ``possible perfection'' notion of traditional software as pointed to by Rushby and Littlewood \cite{littlewood_reasoning_2012,rushby_software_2009}. That is, assurance activities, e.g., performed in support of DO-178C, can be best understood as developing evidence of possible perfection.
%-- a confidence in $\mathit{pfd}=0$. 
Similarly, for safety critical machine learning model, we believe its lifecycle activities should be considered as aiming
to train a ``possibly perfect'' model in terms of the two \textit{reducible} errors. Thus, we may have some confidence that the two errors are both 0 (equivalently, a prior confidence in the \textit{irreducible} Bayes error since the other two are 0), which indeed is supported by on-going research into 
finding globally optimised DNNs \cite{du_gradient_2018}. 
%Meanwhile, on the \textbf{trained model}, V\&V also provides prior knowledge as shown in Example~\ref{example_robustness} below, and \textbf{online monitoring} continuously validates the assumptions for the prior knowledge being obtained.




\subsection*{Neural Architecture Search}

While it is often regarded as dark-art for the  model construction and training because they are usually dependent on the experience of the developer, it is not hard from the above discussion that they can be treated as an optimisation problem to reduce the approximation and estimation errors. 
%
Towards this, neural architecture search \cite{DBLP:journals/corr/ZophL16} has been recently discussed through e.g., 
reinforcement learning to automatically generate high-performing neural network architectures for a given learning task. 
%The learning agent is trained to sequentially choose CNN layers using Q-learning with an $\epsilon$-greedy exploration strategy and experience replay. 
The reinforcement learning agent explores a large but finite space of possible architectures and iteratively discovers designs with improved performance on the learning task. 


\subsection*{Best Practice} 

Neural architecture search requires significant computational resources so may not be applicable to all developments. An alternative way is to follow the best practice in machine learning development, which includes the following steps: 
\begin{enumerate}
    \item Start with a simple model with limited complexity (e.g., small number of features and restricted model capacity). Repeatedly do the following: 
    \begin{enumerate}
        \item tune the hyper-parameters to train a model, and 
        \item apply debugging tools (e.g., verification and validation tools for safety properties, and the evaluation methods as introduced in Chapter~\ref{sec:usualevaluation}) to understand the performance, 
    \end{enumerate}  
    until finding a valid model. Take this model as a baseline model. 
    \item Repeatedly do the following: 
    \begin{enumerate}
        \item add complexity (by e.g., increasing the number of features, and increasing model capacity) to a baseline model, and 
        \item Repeat step 1a and 1b until obtaining a new trained model that performs better than the baseline model. We need to make sure than a more complex model should always perform better than a less complex model. Make the new model as a baseline model.     
    \end{enumerate}
    Note that, the above step 2 may lead to multiple baselines, and we always need to justify the benefit of added complexity with improved performance. 
    \item Select a model from a set of baseline models according to the required, acceptable level of performance. 
\end{enumerate}

It is not hard to see that, the above best practice is also a controlled optimisation process to construct a more and more complex model by gradually increasing the model complexity. Different from the neural architecture search, this process has human in the loop, where the developer needs to design the added complexity and tune the hyper-parameters. 


\subsection*{Expected Outcome}

The outcome of the optimised model construction and training is a well trained model with minimum generalisation estimation and approximation error. The errors can be estimated in an empirical way with a test dataset or some more principled methods such as complexity measures~\cite{jin2020does}. In addition, as indicated in Part~\ref{chap:advtraining}, there might be other requirements, including the safety properties we discussed in Section~\ref{sec:defsafetyissues}, that are needed to be considered in model construction and training. Objective measurements are needed to determine if they are properly implemented.   

\section{Adequacy of Verification and Validation}

This section discusses when a verification and testing technique is able to conclude the sufficiency, or completeness, of the analysis. For example, when using testing method, we need to know when to terminate the test case generation process. When applying robustness verification, we need to know how many local instances we need to verify. We remark that, the discussion in this section is mainly concerned with the high-level safety assurance, considering the operational use of a machine learning model that may have a set of different inputs from a data distribution $\mathcal{D}_{op}$ (see similarly in Section~\ref{sec:assuranceCNN}). It is different from the robustness verification, in which as discussed in \cite{HUANG2020100270} and Part~\ref{chap:verification} the completeness is mainly concerned about the exhaustiveness of the input instances in a small neighbourhood around a given input instance. 

There are mainly two approaches that can be utilised, including the behaviours of a machine learning model in inference stage and the data instances that might appear in operational stage, respectively. For these approaches, objective metrics are needed to determine the extent to which an analysis technique has conducted, as discussed in the later ALARP principle.   

\subsection*{Coverage of Machine Learning Behaviours}

While usually treated as ``black-box'', most machine learning models have internal behaviours when processing a data instance. It has been noted in \cite{9451178} that, even for the complex recurrent neural networks, two input instances with the same internal behaviours, defined as the temporal evolutions of the joint latent representation of all gates and internal states, will represent the same instance and get the same classification result. For convolutional neural networks, the internal behaviours can be the vectors of latent representations of different layers. Therefore, the exploration of all internal behaviours will be adequate for the verification and testing. However, due to the continuous nature, the number of behaviours can be infinite, which suggests that some level of abstractions are usually needed to define behaviours. 


The abstracted behaviours include low-level ones, such as the activation of individual ReLU neurons and the activation of causality relation between neurons, and high-level ones, such as the semantics relations between activation vectors of different layers. The low-level ones have led to the proposals of various structural coverage metrics, such as neuron coverage and MC/DC coverage as we discussed in Section~\ref{sec:test-criteria}, while the high-level ones have led to other proposals, such as the semantics abstraction of the neural networks as in Chapter~\ref{chap:NNabstraction} and the symbolic representation of the temporal evolution of the latent representations as in \cite{9451178}. For the latter, once a semantics representation is defined, the metric is to measure the percentage of possible concrete semantics instances that have been explored for the verification and testing. 

We remark that, for both low-level and high-level behaviours, their metrics might not be able to reach 100\% coverage, because some behaviours can be infeasible for the machine learning model. Therefore, instead of setting up threshold for the termination of analysis, empirical methods, such as a similar technique as the learning curve as discussed in Section~\ref{sec:trainingsufficiency}, will be needed to determine whether the analysis has been adequate with respect to the  metric. 

A more intriguing observation is that, these metrics might not be tight enough to study a machine learning model, for example, certain neuron activation, defined as a behaviour of a convolutional neural network processing images, may appear multiple times when a neural network works with different input instances. This is mainly due to the fact that the abstracted behaviour is too coarse.  Therefore, the definition of behaviours need to be carefully designed so that it strikes a balance between adequacy and complexity.  

\subsection*{Coverage of Operational Use} 

An alternative way of considering the adequacy of verification and testing is to explore the set of all possible input instances. That is, if we are able to enumerate all instances that may appear when the machine learning model is used, and confirm their safety, we are certain about the safety of the machine learning model.  

However, due to the missing of the true data distribution $\mathcal{D}$, it is unlikely that we are able to directly enumerate the operational data instances. To deal with this, methods such as variational autoencoder and generative adversarial network can be utilised to learn the data distribution. Based on the learned data distribution, a set of seeds can be selected for the analysis, as discussed in~\cite{DBLP:journals/corr/abs-2112-00646}. 

Moreover, the direct working with data distribution  $\mathcal{D}$ enables the possibility of integrating human prior knowledge. Intuitively, human experts may have a good level of knowledge that certain features can be more important (and therefore should be weighted higher) in an operational environment than another. For example, snowing on images can be more relevant for Toronto than California. Such prior knowledge can be integrated into the determination of the coverage of operational use (e.g., the learned data distribution) so that the resulting verification and validation is contextually relevant. 

\subsection*{ALARP (``As Low As Reasonably Practicable")}

For both the above coverage methods, the continuity and high-dimensionality of the spaces to be covered make the exhaustive enumeration unlikely. Therefore, certain adaptation of the ALARP (``As Low As Reasonably Practicable") principle might be helpful to strike a cost-benefit balance. Principled approaches are needed to weight the safety risks and measure the cost needed to identify the risks. Then, a monitoring process runs in parallel with the verification and validation process to determine when the cost involved in identifying the risks would be grossly disproportionate to the benefit gained. 


\subsection*{Expected Outcome} 

The outcome of the adequacy of verification and validation is either a proof of the adequacy or a validation report containing objective measurement on the adequacy. A justification on the cost-benefit balance is also needed. 

\section{Assured Partnership of AI and Humans}\label{sec:assuredPartnership}

Up to now, all the topics are based on data, algorithms, and models, with the interactions with humans mainly on requiring humans to e.g., label the training instances, and differentiate whether a perturbed image is an adversarial example. These interactions completely rely on humans' functional ability, without utilising their social aspects of intelligence. However, AI systems, or software/hardware systems with AI components, are penetrating our everyday life. It is therefore imperative to consider the partnership of humans and AI. 
From safety perspective, when humans and AI are interacting, the safety issues become more prominent, because unexpected behaviours of the AI systems may easily lead to bad consequences on their human partners.  

In this section, we will discuss several technical topics on how to assure the partnership of AI with humans. They include requiring a machine learning agent to provide auxiliary information beyond decisions, such as the confidence of each decision (i.e., \emph{uncertainty estimation}) and the explanation of individual decisions and the model in general (i.e., \emph{explainability and interpretability}), and understand human context (i.e., \emph{contextualisation}). These technical means can help strengthen the trust between humans and AI, and therefore enable their safe co-operation. 

\subsection*{Uncertainty Estimation}



Essentially, a machine learning model is  a function that is obtained by minimising a given loss function over a set of training data. The function may behave wrongly, which can be evaluated with verification and validation techniques or detected with runtime monitoring techniques. In addition to the misbehaviour, another key concern is on the confidence of the machine learning model in making the wrong decision. It is clear that the situation is worse if a misbehaviour is conducted with high confidence than with low confidence. The confidence naturally comes with the Bayesian view on learning, where a deterministic trained model (such as a trained neural network) is seen as a sample from a distribution of models. Uncertainty estimation on machine learning models has been intensively discussed, with some existing methods such as MC dropout \cite{10.5555/3045390.3045502} and deep ensemble~\cite{DBLP:conf/nips/Lakshminarayanan17} that can be applied for neural networks. 

The evaluation of uncertainty estimation methods is non-trivial, mainly due to the lack of ``ground-truth'' uncertainties. It is therefore useful to evaluate against a set of concrete baseline datasets and evaluation metrics that cover all types of uncertainties. In addition, a typical measurement regarding risk-averse and worst case scenarios is usually considered. Specifically, it requires that uncertainty predictions with a very high predicted uncertainty should never fail. 


\subsection*{Explainability and Interpretability}\label{sec:explainableAI}

To gain trust from human users in operational use, it is essential to enable them to understand the decisions a machine learning model has made. Explainable AI is a topic with fierce discussions lately. The reader is referred to recent surveys such as \cite{HUANG2020100270} for more explanations on the recent progress in this direction. There are a number of principles that have been frequently used to evaluate whether an explanation is good, such as accuracy, fidelity, consistency, stability, comprehensibility, certainty, and degree of importance \cite{DBLP:journals/corr/abs-1911-10104}.  

\subsection*{Contextualisation}\label{sec:contextualAI}




Contextual AI is a collection of techniques aiming to embed human context into AI systems so as to enable their interaction with humans. First of all, context-awareness requires that the AI systems are able to ``see'' at the same level as a human does. Then, based on the \emph{human-level observation}, contextual AI is capable of analysing the cultural, historical, and situational aspects surrounding incoming data, and synthesising a context that makes the most sense to the humans. Such \emph{human-level reasoning} enables the contextual AI to have the sufficient understanding about the human’s environment, situation, and context. 
It is based on this level of understanding that it is able to explain, reason, behave, and collaborate with the human. 
%


Unfortunately, neither the learning algorithms nor the analysis techniques naturally have human-level observation and reasoning. For example, for image classification task, both the deep learning algorithms and their analysis techniques are primarily based on pixels, while humans understand the images through high-level features. Typical ways of enhancing machine learning algorithms with contextualisation can be done through e.g., apply explainable AI techniques to obtain human-level observation, synthesise context into structures such as knowledge graph and Bayesian network, and then 
conduct human-level reasoning (e.g., logic reasoning, commonsense reasoning, or probabilistic inference) over the synthesised structures. 


Analysis techniques, such as verification and validation techniques, should also been lifted to human-level observation and reasoning. Actually, the consideration of pixel-level analysis techniques has led to the notorious scalability issues that verification techniques are only able to work with either small-size neural networks (as in Chapter~\ref{chap:MILP}) or limited number of input dimensions (as in Chapter~\ref{chap:reachabilityAnalysis}). Even for testing techniques, the scalability is also an issue when considering tighter coverage metrics such as MC/DC (Section~\ref{sec:test-criteria}). Therefore, the analysis techniques to support the reasoning on higher-level features are needed to not only make the analysis make sense to humans but also focus the limited analysis cost on the most important aspects. 

Another perspective is on the risk of de-contextualisation in terms of the choice of models. Models originally used for one purpose may not be suitably re-used in a different context and for a different purpose. Validation activities are required to understand the impact of contextual changes on the safety. 

\subsection*{Expected Outcome}

The outcome of assured partnership is a machine learning model that is able to perceive, reason, and behave at the same level as its human partners. The three perspectives discussed above (uncertainty estimation, explainability and interpretability, and contextualisation) are essential for this purpose. In addition to their respective evaluations in particular for the uncertainty estimation and explainability and interpretability, a holistic evaluation on the assured partnership, or more formally trust between humans and AI, should also be considered. The trust evaluation is needed to be supported by rigorous reasoning frameworks such as \cite{10.1145/3329123},  where the trust is quantitatively measured, and dynamically updated with the interactions, to enable verification techniques to be applied.
Empirical experiments based on these theoretical frameworks should be conducted to validate the success of partnership. 



\chapter{Probabilistic Graph Models for Feature Robustness}\label{chap:pgmfeature}

%\newpage
%\section{Introduction}

Up to now, we have known that machine learning algorithms can be used to effectively learn a function $f$ from a set of input-output pairs. The function $f$ approximates the relation between two random variables $X$ and $Y$, and actually expresses the conditional probability $P_f(Y|X)$. However, in a complex, real world system, there might be more than two random variables and it is useful to not only understand the the conditional probability between random variables but also be able to infer more intriguing information from the conditional dependence relations between random variables.
It is also possible that, a complex machine learning model, such as a convolutional neural network, can be approximated by constructing the conditional dependencies between a set of random variables representing the features learned by the neural networks, see e.g., \cite{berthier2021abstraction} for an example.  Therefore, while it is agreeable that machine learning has been able to support human operators in dealing with some long-standing tasks such as object detection and recognition with accuracy and efficiency, there is still a need to infer useful knowledge from a set of conditional probabilities. 
This chapter is to explain how we may utilise probabilistic graphical model, a formalism to express conditional probabilities between random variables ($X$, $Y$, and variables for latent representations), as an abstract model for neural network. We will present the definition in Section~\ref{sec:defPGM} and then discuss the abstraction method in Section~\ref{chap:NNabstraction}. More detailed introduction to probabilistic graphical models is given in the Appendix (Chapter~\ref{chap:pgm}). 

\section{Definition of Probabilistic Graphical Models}\label{sec:defPGM}

Probabilistic graphical models are a formalism for the above purpose. They use a structure, or more specifically a graph, to represent conditional dependence relations between random variables, and use a probability table for every random variable to express the local dependence relation of the random variable. There are two major branches of graphical models, namely, Bayesian networks and Markov random fields, and in this chapter, we focus on Bayesian network. In the following, we will use (probabilistic) graphical models and Bayesian network interchangeably. 


Depending on the dependence relation of individual random variable, the probability tables in a graphical model can be a marginal probability table, which shows that the random variable does not depend on any other variables in the graph, or a conditional probability table, which represents the conditional probability distribution of the current random variable over other random variables in the graph. 

\begin{figure}[!htbp]
    \centering
    \includegraphics[width=0.7\textwidth]{images/graphical models/threeNodes.png}
    \caption{A simple graphical model of three nodes}
    \label{fig:threeNodes}
\end{figure}


Figure~\ref{fig:threeNodes} presents a simple graphical model with 3 nodes: $I, G, S$. We can see that one of the nodes $I$ has a marginal probability table while the remaining two nodes have conditional probability tables. 
Summarising, 
\begin{equation}
    \text{Probabilistic Graphical Model = Graphical Structure + Multivariate Statistics}
\end{equation}
Formally, a probabilistic graphical model $G=({\cal V},{\cal E},P)$ where ${\cal V}$ is a set of nodes, representing the random variables, ${\cal E}$ is the set of edges between nodes, and $P$ is a set of probability tables, one for each node in ${\cal V}$. 

\section*{A Running Example}

Assume that, on a self-driving car, there are two sensors, $Camera$ and $Radar$, that are used to detect pedestrian collectively. The precision of the  camera may be affected by weather conditions, such as the $Fog$ as we consider in this example. The $Radar$ may be affected by the distance of the object from the car, i.e., it can be very precise when the object is close but may become less precise when the object is $Away$. Once a pedestrian is detected and it is not away, the car will need to stop. 

%\begin{example}


\begin{figure}[!htbp]
    \centering
    \includegraphics[width=0.6\textwidth]{images/graphical models/graphicalmodel.png}
    \caption{A simple Bayesian network for safety analysis on vehicle stopping upon pedestrian detection}
    \label{fig:graphicalmodel}
\end{figure}

Figure~\ref{fig:graphicalmodel} presents a probabilistic graphical model for this example. In the graph $G$, there are six random variables: $Camera$, $Radar$, $Fog$, $Detected$, $Away$, and $Stopped$. Every node is associated with either a marginal probability table or a conditional probability table, depending on whether they have incoming edges. The information about the probability tables are given in Figure~\ref{fig:cameratables}. For example, the nodes $Camera$ and $Radar$ do not have incoming edges, so each of them is associated with a marginal probability table. Intuitively, the two tables suggest that the probability of a pedestrian appearing in the imagery input of the camera is $0.4$, and in the signal input of the radar is $0.5$. Note that, the ``appearing'' is for ground truth (through human's eyes), not for the result of a detection system. The detection is implemented through the $Detected$ node to be explained below. 


%\end{example}


\begin{figure}[!htbp]
    \centering
    \includegraphics[width=\textwidth]{images/graphical models/cameramodel.png}
    \caption{Probabilistic table of the graphical model in Figure~\ref{fig:graphicalmodel}}
    \label{fig:cameratables}
\end{figure}


Other nodes are associated with conditional probability tables. For example, the table for $Fog$ shows that, when there is no pedestrian appearing in the imagery input, the probability of the foggy weather condition is $0.5$.  This probability is lowered when there is a pedestrian appearing in the imagery input. This is intuitive, because the foggy condition may affect the ability of camera capturing the pedestrian. Similar for the $Away$ node. When there is no pedestrian appearing in the signal input, the probability of its away from a pedestrian is $0.8$. This probability is lowered to $0.4$ when there is a pedestrian appearing in the signal input. 

The detection result is a fusion of both camera and radar's results. Note that, even if a pedestrian appears in the imagery input, it does not mean that the pedestrian can be detected (Recall the generalisation error and robustness error of deep learning). We note that, if neither of the sensors has a pedestrian appeared, no detection can be made at all. If one of the sensors has a pedestrian, there is a non-trivial chance that it can be detected. The detection becomes significantly better when both sensors captured the pedestrian. 

Finally, the decision making on whether the car should be stopped is based on both the detection result and the distance. If it is detected (i.e., $Detected = d_1$) and not far away (i.e., $Away = a_0$) then this probability is high ($0.9$). Otherwise, the probability is low ($0.1$). The lowest probability appears when no pedestrian is detected (i.e., $Detected = d_0$) and it is away (i.e., $Away = a_1$). 

\subsection*{Where does machine learning play a role?}

Machine learning can be used to generate those conditional probability tables. For example, a deep learning model can be designed and trained to get the table for $Detected$ node, i.e., classify whether a pedestrian is detected or not on both the camera input and the radar input. Similarly, other nodes such as $Fog$, $Away$, and $Stopped$ may also be implemented with a machine learning model. 



%\newpage
\section{Abstraction of Neural Network as Probabilistic Graphical Model}\label{chap:NNabstraction}


In this section, we construct a Bayesian network out of a trained neural network. In the end, the Bayesian network captures the distribution of neuron valuations in terms of latent features encoded in each neural network layers, as well as their causal relationships.

\subsection{Extraction of Hidden Features}\label{sec:featureextraction}

Assume that some feature extraction technique such as PCA and ICA has been used to analyse the neuron activation vector $\textbf{v}_i$ of layer $i$ that are induced by  a given \emph{training set} \(D_{train}\).
This produces a set of feature mappings $\Features{i}=\{\lambda_{i,j}\}_{j \in \{1,…,t_i\}}$ for $t_i$ features, such that each $(\lambda_{i,j}: \layerdom{i} \rightarrow \featcomp{i}{j})$ maps the vector space $\layerdom i$ 
of neuron valuation into the $j$-th component of the feature space $\featdom_{i,j}$.
%The latter space is the product \(\featdom i ≝ \prod_{j\in{1,…,s_i}}{\featcomp i j}\mbox.\)

Actually, \Features{i} is such that the neuron values $\textbf{v}_i$ for any input \(\textbf{x} \in D_{train}\), can be transformed into a \mbox{\(t_i\)-}dimensional vector
\begin{equation}
  \langle\lambda_{i,1}\textbf{v}_i,…,\lambda_{i,t_i}\textbf{v}_i\rangle \in \featdom i
\end{equation}
where $\lambda_{i,j}\textbf{v}_i$ represents the \mbox{\(j\)-}th component of the value obtained after mapping the instance $\textbf{x}$ into the feature space.
%We will refer to the projection \(λ_{i, j}∘\textbf{v}_i(x)\) as the \emph{latent feature valuation induced by $\textbf{x}$ on component \featcomp i j}.
%
%\begin{leaveout}
%  In the context of a layer $i$ where values output by all neurons result from the application of a Rectified Linear Unit (ReLU), we will further identify the projection of the \emph{activation threshold} into the feature space as \(λ_{i}(0) ≝ 〈λ_{i,1}(0), …, λ_{i,t_i}(0)〉\mbox.\)
%\end{leaveout}
%
\begin{figure}[t!]
    \centering
    \includegraphics[width=\textwidth]{images/graphical models/diagram.png}
    \caption{Reducing Neural Networks to Bayesian Networks}
    \label{fig:diagram}
\end{figure}
%
Figure~\ref{fig:diagram} gives an illustrative diagram of reducing $\textbf{v}_1, \textbf{v}_2,\textbf{v}_3$ to features.
In particular, each $\textbf{v}_i$ is reduced to two features $\lambda_{i,1}\textbf{v}_i$ and $\lambda_{i,2}\textbf{v}_i$.


\subsection{Discretisation of Hidden Feature Space}\label{sec:discretisation}

The feature extraction techniques 
%mentioned in Section~\ref{sec:dimens-reduct-via} 
result in mappings $\lambda_{i,j}% \textbf{v}_i
$ that range over a continuous and potentially infinite domain.
%, such as ℝ.
Yet, Bayesian network-based abstraction technique relies on the construction of \emph{probability tables}, where each entry associates a set of \emph{distinct} latent feature values with a probability.
For this construction to be relevant, we therefore \emph{discretise} each latent feature component into a \emph{finite} set of sub-spaces.

%

\subsection{Construction of Bayesian Network Abstraction}
\label{sec:constr-bayes-netw}


The abstraction that we construct primarily represents the \emph{probabilistic distribution} of the set of latent feature values induced by a set $\textbf{X}$ of test instances.
In other words, given an input \(\textbf{x} \in 
\textbf{X}\), the abstraction allows us to estimate the probability that $\textbf{x}$ induces a given combination of values for the latent features that have been learned by the neural network.

Thanks to the layered and a\-cyclic nature of the neural networks that we consider, we can directly characterise the \emph{causal relationship} between the sets of neuron values in various layers \wrt a series of inputs as well.
In other words, given an input \(\textbf{x} \in 
\textbf{X}\), one can in principle estimate the conditional probability of each neuron value at layer $i$ \wrt the probability of every combination of neuron values at layer $i-1$.
By lifting the above relationship from individual neuron values to latent feature intervals, we seek to capture \emph{causal semantic relations} that link the features at each layer: in a layer $i$, and with an input $\textbf{x}$, the \emph{probability} that a latent feature valuation belongs to a given interval in the corresponding feature space is \emph{dependent} on probabilities pertained to latent feature intervals at layer $i-1$.



\subsection{Preserved Property}

First of all, we show that the constructed Bayesian network is an abstraction of the neural network.
Given a finite set $\textbf{X}$ of inputs, an abstraction constructs a set $\textbf{X}' \supseteq \textbf{X}$ that generalises $\textbf{X}$ to more elements \cite{DBLP:journals/corr/abs-1911-09032}. Given an input $\textbf{x}$ and a Bayesian network $\lambda(\textbf{X})$, we are able to check the probability of $\textbf{x}$ on $\lambda(\textbf{X})$, i.e., $\lambda(\textbf{X})(\textbf{x})$. We say that $\textbf{x}$ is included in $\lambda(\textbf{X})$ if $\lambda(\textbf{X})(\textbf{x})>0$.
The following lemma suggests that every sample in $\textbf{X}$ is included in $\lambda(\textbf{X})$:

\begin{lemma}\label{lemma:abstraction}
  All inputs $\textbf{x}$ in the dataset $\textbf{X}$ are included in the $\lambda(\textbf{X})$ with probability greater than 0.
\end{lemma}

Let $\textbf{X}'$ be the set of inputs that satisfy $\lambda(\textbf{X})(\textbf{x})>0$. This lemma suggests that $\textbf{X}'\supset \textbf{X}$. Therefore, $\lambda(\textbf{X})$ defines an abstraction of the dataset $\textbf{X}$.
This abstraction also suggests that the abstraction assumption -- \ie inputs that are outliers \wrt the abstraction are also outliers \wrt the original neural network -- is reasonable because $\textbf{X}'\supset \textbf{X}$.

In addition to this simple property, \cite{berthier2021abstraction,Alshareef2022} also consider other analysis techniques based on the abstracted Bayesian network. 



\Extrachap{Exercises}
%\addcontentsline{toc}{chapter}{\protect\numberline{}Exercise}%

\begin{newquestion}{\textbf{1}~~}
Write a PCTL formula to express that the agent will never move into certain states labelled with atomic proposition $p$. 
\end{newquestion}

\begin{newquestion}{\textbf{2}~~}
Write a program to automatically construct a failure process DTMC according to a set of trajectories. 
\end{newquestion}


\begin{newquestion}{\textbf{3}~~}
What is the size of model ${\cal M}^E(\pi,\textbf{x}_0, \mathcal{C})$, compared with ${\cal M}^E(\pi,\textbf{x}_0)$? 
\end{newquestion}


\begin{newquestion}{\textbf{4}~~}
Can you identify a test coverage metric that is different from the ones presented in Chapter~\ref{chap:testing}? 
\end{newquestion}


\begin{newquestion}{\textbf{5}~~}
What is the relation between reliability and robustness? 
\end{newquestion}

\begin{newquestion}{\textbf{6}~~}
Please give a list of assurance techniques with respect to the lifecycle stages of machine learning model. 
\end{newquestion}

\begin{newquestion}{\textbf{7}~~}
Can we use the abstracted Bayesian network as described in Chapter~\ref{chap:pgmfeature} for the prediction? If so, what do you think of its accuracy, when comparing with the accuracy of the original network? 
\end{newquestion}
