%%%%%%%%%%%%%%%%%%%%%%preface.tex%%%%%%%%%%%%%%%%%%%%%%%%%%%%%%%%%%%%%%%%%
% sample preface
%
% Use this file as a template for your own input.
%
%%%%%%%%%%%%%%%%%%%%%%%% Springer %%%%%%%%%%%%%%%%%%%%%%%%%%

\preface




This book addresses the safety and security perspective of machine learning, focusing on its vulnerability to environmental noise and various safety and security attacks. Machine learning has achieved human-level intelligence in long-standing tasks such as image classification, game playing, and natural language processing. However, like other complex software systems, it is not without any shortcomings, and a number of hidden issues have been identified in the past years. The vulnerability of machine learning has become a major roadblock to the deployment of machine learning in safety-critical applications. 


We will first cover falsification techniques to  identify the safety vulnerabilities on various machine learning models, and then devolve them into different solutions to evaluate, verify, and reduce the vulnerabilities. The falsification is mainly done through various attacks such as robustness attacks, data poisoning attacks, etc. Compared with the popularity of attacks, solutions are less mature, and we consider solutions that have been broadly discussed and recognised (such as formal verification, adversarial training, and privacy enhancement), together with several new directions (such as testing, safety assurance, and reliability assessment). 

%verify the trained machine learning models, and enhance the training process to reduce the vulnerabilities. 





%This will be a new textbook for undergraduate students, covering contents on both machine learning and their safety and reliability issues. Both theoretical and practical elements are included. 

%Unlike the usual machine learning books which concentrate on how to train a good machine learning model, this book includes other contents that are related to their safety and reliability issues, which need to be seriously considered before applying the machine learning models to practical applications. While the safety and reliability issues have been intensively studied in research community, they have not been made aware to undergraduate students. With the support of practical contents, this book will help the undergraduate students understand the theoretical contents but also know the current techniques on dealing with these issues. 



Specifically, this book includes four technical parts. Part \ref{chap:intro} introduces basic concepts of machine learning, as well as the definitions of its safety and security issues. This is followed by the introduction of techniques to identify the safety and security issues in  machine learning models (including both transitional machine learning models and deep learning models) in Part~\ref{chap:simple}. Then, we present in Part~\ref{chap:verification} two categories of safety solutions that can verify (i.e., determine with provable guarantees) the robustness of deep learning  and that can enhance the robustness, generalisation, and privacy of deep learning. In Part~\ref{chap:lookfurther},
%we consider probabilistic graphical model and how it can be applied to reason about the latent features of deep learning. Finally, 
we discuss several extended safety solutions that consider either other machine learning models or other safety assurance techniques. We also include technical appendices. 

%a few topics that are not covered in the previous chapters but are no less important.


The book aims to improve the awareness of the readers, who are future developers of machine learning models, on the potential safety and security issues of machine learning models. More importantly, it includes up-to-date content regarding the safety solutions for dealing with safety and security issues. While these solution techniques are not sufficiently mature by now, we are expecting that they can be further developed, or can inspire new ideas and solutions, towards the ultimate goal of making machine learning safe. We hope this book can pave the way for the readers to become researchers and leaders in this new area of machine learning safety, and the readers will not only learn technical knowledge 
but also gain hands-on practical skills. Some source codes and teaching materials are made available at
\begin{quote}
    \url{https://github.com/xiaoweih/AISafetyLectureNotes}. 
\end{quote}
%on  %on 
%how to develop machine learning algorithms for a given dataset. In addition to this, they will be able to understand not only 
%when and how 
%how an excellent machine learning may perform badly, 
%in unexpected scenarios. 
%Together with the awareness of safety and reliability issues, we also provide techniques to 
%and how to verify and enhance the machine learning models. 

%As we are currently practicing at the University of Liverpool, the students are expected to have some basic programming skills on Python. Other than this, we expect them to have some basic understanding about artificial intelligence. 

 

\vspace{\baselineskip}
\begin{flushright}\noindent
London/Liverpool/Exeter, UK,\hfill {\it Xiaowei Huang}\\
March 2022\hfill {\it Gaojie Jin}\\
\hfill {\it Wenjie Ruan}\\
\end{flushright}


