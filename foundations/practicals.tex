\newpage 
\chapter{Practice}\label{sec:foundationspracticals}

In this chapter, we will explain how to install an experimental environment based on Python, and present a few examples of basic Python operations and the code for the visualisation of a simple dataset (as in Example~\ref{example:3dscatter}) and the confusion matrix (as introduced in Section~\ref{sec:confusionmatrix}). 

\section{Installation of Experimental Environment}

%\subsection*{Hardware/software environment} 
First of all, your machine needs to have Python3 or Anaconda installed. Then, to install the python packages that will be used later, you have the following two options. 
%

\subsection*{Use Pip} 
The first option is to use Pip for installation. 
First, you need to make sure that python and pip (or pip3) are installed appropriately. You can check this with the `pip -V' in Windows Commands (cmd) or MacOS Terminal. Then, the installation of software packages is done through the following example: 
\begin{cmds}
pip (or pip3) install matplotlib
\end{cmds}

\subsection*{Create Conda Virtual Environment}
You can download the Anaconda through  \url{https://www.anaconda.com/products/individual}, create an environment and install some necessary software packages: 

\begin{cmds}
conda create --name aisafety
conda activate aisafety
conda install pandas numpy matplotlib tensorflow scikit-learn pytorch torchvision
\end{cmds}

In addition to the installation of packages, you could install Jupyter notebook through \url{https://jupyter.org/install}, for the editing and running of python code via a web browser. 

\subsection*{Test Installation}

Once installed, you can check whether the installation is successful by running the following commands. Note that, the first command is to activate the Python environment, in which the remaining commands are executed.  

\begin{cmds}
python3
import numpy
import scipy
import sklearn
import matplotlib
import pandas
exit()
\end{cmds}

Once the installation is successful, create a new file \textbf{lab1.py} and type in the following lines: 

\begin{lstlisting}[language=Python]
import numpy as np

x = np.arange(100)
y = np.array(5)

z=x+y
\end{lstlisting}

%You need to understand the meaning of the command. 
Then, you can use the following command to check the result: 
\begin{cmds}
python (or python3) lab1.py
\end{cmds}

\section{Basic Python Operations}

In the following, we provide a few exercises for basic Python operations. Please complete them sequentially. 

\begin{enumerate}
    \item Using array indexing to give the last ten values of z
    
\begin{lstlisting}[language=Python]
xLastTen = x[90:] # Or x[-10:]
\end{lstlisting}

\item Update the code so that x goes from 0 - 1000 in steps of 10

\begin{lstlisting}[language=Python]
xUpdate = np.arange(0, 1000, 10)
\end{lstlisting}

\item Take dot product between x and x

\begin{lstlisting}[language=Python]
xDotProduct = x.dot(x)
\end{lstlisting}

\item Take (*) product between x and x

\begin{lstlisting}[language=Python]
xAsteriskProduct = x * x
\end{lstlisting}

\item Reshape x so that it is no longer a 100 value array but a 10x10 matrix

\begin{lstlisting}[language=Python]
xReshape = xUpdate.reshape((10, 10))
\end{lstlisting}

\item Multiply the first row by 1, the second by 2, the third by 3 and so on

\begin{lstlisting}[language=Python]
yNew = np.arange(1,11)
zNew = xReshape * yNew[:, np.newaxis]
print(zNew)
\end{lstlisting}

\item Using the matplotlib library to plot each row of this matrix as a single series on the same graph

\begin{lstlisting}[language=Python]
for i in range(10):
   plt.plot(zNew[i])
plt.show()
\end{lstlisting}

\item Using the matplotlib library to plot each row of this matrix as a single series on separate sub-plots of the same figure and save this figure as figure1.png

\begin{lstlisting}[language=Python]
for i in range(10):
    ax = plt.subplot(5, 2, i + 1)
    plt.plot(zNew[i])
plt.savefig('figure1.png')
plt.show()
\end{lstlisting}

\end{enumerate}



\section{Visualising a Synthetic Dataset}

Below is the code for the visualisation of the synthetic dataset as given in Example~\ref{example:3dscatter}. 

\begin{lstlisting}[language=Python]
import numpy as np
import matplotlib.pyplot as plt

xdata = 7 * np.random.random(100)
ydata = np.sin(xdata) + 0.25 * np.random.random(100)
zdata = np.exp(xdata) + 0.25 * np.random.random(100)

fig = plt.figure(figsize=(9, 6))
# Create 3D container
ax = plt.axes(projection = '3d')
# Visualize 3D scatter plot
ax.scatter3D(xdata, ydata, zdata)
# Give labels
ax.set_xlabel('x')
ax.set_ylabel('y')
ax.set_zlabel('z')
# Save figure
plt.savefig('3d_scatter.png', dpi = 300);
\end{lstlisting}

To run this code,
%(in the file ``1.4.1.txt''), 
you need to activate the environment you have created earlier, with the following command: 
\begin{cmds}
conda activate aisafety
python3 4.3.txt
\end{cmds}
This will be needed for all future practicals. 



\section{Confusion Matrix}

We train a simple perceptron model and output the confusion matrix. 

\begin{lstlisting}[language=Python]
from sklearn import datasets
dataset = datasets.load_digits()
X = dataset.data
y = dataset.target

print("===== Get Basic Information ======")
observations = len(X)
features = len(dataset.feature_names)
classes = len(dataset.target_names)
print("Number of Observations: " + str(observations))
print("Number of Features: " + str(features))
print("Number of Classes: " + str(classes))

print("===== Split Dataset ======")
from sklearn.model_selection import train_test_split
X_train, X_test, y_train, y_test = train_test_split(X, y, test_size=0.20)

print("===== Model Training ======")
from sklearn.linear_model import Perceptron
clf = Perceptron(tol=1e-3, random_state=0)
clf.fit(X_train, y_train)

print("===== Model Prediction ======")
print("Labels of all instances:\n%s"%y_test)
y_pred = clf.predict(X_test)
print("Predictive outputs of all instances:\n%s"%y_pred)

print("===== Confusion Matrix ======")
from sklearn.metrics import classification_report, confusion_matrix
print("Confusion Matrix:\n%s"%confusion_matrix(y_test, y_pred))
print("Classification Report:\n%s"%classification_report(y_test, y_pred))
\end{lstlisting}