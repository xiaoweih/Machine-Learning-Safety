%\newpage 
\chapter{Probability Theory}

\section{Random Variables}

In most of our contexts, a random variable is a function that assigns probability values to each of an experiment's (or an event's) outcomes. Intuitively, a random variable has a probability distribution, which represents the likelihood that any of the possible values would occur.

\begin{example}\label{example:foundationsgrades}
We have a population of students, such that we want to reason about their grades. In this case, we have 
\begin{itemize}
    \item a random variable $Grade$, with the set of possible values $V(Grade)=\{ A, B, C \}$, and 
    \item a function $P(Grade)$, which associates a probability with each outcome. 
\end{itemize}
\end{example}



Given $P(X)$ is a probability distribution, we have 
\begin{equation}
    \sum_{x\in V(X)} P(x) = 1
\end{equation}
and we may call $P(X)$ the marginal distribution of $X$, as opposed to the joint and conditional distributions. We will explain these concepts in detail later.  

A probability distribution $P(X)$ is a multi-nominal distribution if there are multiple values for the random variable $X$, i.e., $|V(X)|>1$. Moreover, if $V(X)=\{false,true\}$ then $P(X)$ is a Bernoulli distribution. Besides, $P(X)$ can be continuous, such that it may take on an uncountable set of values. 

\begin{example}
The $Grade$ random variable in Example~\ref{example:foundationsgrades} has three possible values and therefore it is associated with a multi-nominal distribution. Also, for a coin tossing example with a single coin that may be either head or tail, the random variable $Coin$ is associated with a Bernoulli distribution. Moreover, a real-valued random variable is continuous, even if it only takes values from a real interval. 
\end{example}


A random variable can also be multi-dimensional. 

\begin{example}
For the \textbf{iris} example as in Example~\ref{example:iris}, the data instance can be seen as a 4-dimensional continuous variable $X$, such that it has an underlying data distribution and the dataset is sampled from the data distribution. The label $Y$ can be seen as another random variable.    
\end{example}

%Such a distribution is also referred to as a multi-nomial distribution as there are multiple values for the random variable $X$. If 

%Also, 

\section{Joint and Conditional Distributions}

A marginal probability is the probability of an event $X$ occurring, i.e., $P(X)$. It may be thought of as an unconditional probability, as it is not conditioned on any other event.

\begin{example}
Assume that we randomly draw a card from a standard deck of playing cards. Let $C$ be the random variable, representing the color of the card drawn. Then, the probability that the card drawn is red is $P(C = red) = 0.5$, or simply $P(red) = 0.5$ if the random variable $C$ is clear from the context.
\end{example}

\begin{example}
Given a standard deck of playing cards, the probability that a card drawn is a 4 is $P(four)=1/13$.
\end{example}
%• Another example: 

\subsection*{Joint probability $P(X,Y)$} 
is the probability of event $X$ and event $Y$ occurring. It is a statistical measure that calculates the likelihood of two events occurring at the same time. %It is the probability of the intersection of two or more events. The probability of the intersection of $A$ and $B$ may be written $P(A \cap B)$.

\begin{example}
Given a standard deck of playing cards, the probability that a card is a four and red, i.e., $P(four, red) = 2/52=1/26$. Note: there are two red fours in a deck of 52, the 4 of hearts and the 4 of diamonds.
\end{example}

The joint probability can be generalised to work with more than two events. 


\subsection*{Conditional probability $P(X|Y)$}
 is the probability of event $X$ occurring, given that event $Y$ has already occurred.

\begin{example}
Given a standard deck of playing cards, if we know that you have drawn a red card, what is the probability that it is a four? The answer is  $P(four|red)=2/26=1/13$, i.e., out of the 26 red cards (given a red card), there are two fours, so $2/26=1/13$.
\end{example}

\section{Independence and Conditional Independence}\label{sec:indepandcondindep}

Without loss of generality, we assume that for the remaining of this section, all random variables are Boolean. 
Given two Boolean random variables $X$ and $Y$, we expect that, in general, $P(X|Y)$ is different from $P(X)$, i.e., the fact that $Y$ is true may  change our probability over $X$.  

\subsection*{Independence}

Sometimes an equality can occur, i.e, $P(X|Y)=P(X)$.  
 i.e., learning that $Y$ occurs does not change our probability of $X$. In this case, we say event $X$ is independent of event $Y$, denoted as %
 \begin{equation}
     X \bot Y
 \end{equation}
%
A distribution $P$ satisfies $X \bot Y$ if and only if $P(X,Y)=P(X)P(Y)$. %A more common situation is when two events are independent given an additional event 


\begin{example}\label{example:independenceexample}
Consider the joint probability table as shown in Table~\ref{tab:simpleJointProbability}. 
\begin{table}[!htbp]
    \centering
    \begin{tabular}{|c|c|c|}
    \hline
        X & Y & P(X,Y)  \\
        \hline
        $0$  &  $0$ & 0.08 \\
        $0$  &  $1$ & 0.32 \\
        $1$  &  $0$ & 0.12 \\
        $1$  &  $1$ & 0.48 \\
        \hline
    \end{tabular}
    \caption{A simple two variable joint distribution}
    \label{tab:simpleJointProbability}
\end{table}

First of all, we note that 
\begin{equation}
    \begin{array}{lcl}
        P(X=0) & = & 0.32+0.8 = 0.4 \\
        P(X=1) & = & 0.6 \\
        P(Y=0) & = & 0.08+0.12 = 0.2 \\
        P(Y=1) & = & 0.8 \\
    \end{array}
\end{equation}
Then, we notice that 
\begin{equation}
    \begin{array}{lclclcl}
      0.08 & = &  P(X=0,Y=0) & = & P(X=0)*P(Y=0) & = & 0.4 * 0.2 \\
      0.32 & = &  P(X=0,Y=1) & = & P(X=0)*P(Y=1) & = & 0.4 * 0.8 \\
      0.12 & = &  P(X=1,Y=0) & = & P(X=1)*P(Y=0) & = & 0.6 * 0.2 \\
      0.48 & = &  P(X=1,Y=1) & = & P(X=1)*P(Y=1) & = & 0.6 * 0.8 \\
    \end{array}
\end{equation}
that is, 
\begin{equation}
    P(X,Y) = P(X) * P(Y)
\end{equation}
which suggests that $X$ and $Y$ are independent.
\end{example}

\begin{example}\label{example:noindependence}
On the other hand, for the following Table~\ref{tab:simpleJointProbability2}, the two variables $X$ and $Y$ are not independent.  

\begin{table}[!htbp]
    \centering
    \begin{tabular}{|c|c|c|}
    \hline
        X & Y & P(X,Y)  \\
        \hline
        $0$  &  $0$ & 0.10 \\
        $0$  &  $1$ & 0.16 \\
        $1$  &  $0$ & 0.64 \\
        $1$  &  $1$ & 0.10 \\
        \hline
    \end{tabular}
    \caption{Another simple two variable joint distribution}
    \label{tab:simpleJointProbability2}
\end{table}
\end{example}


\subsection*{Conditional independence}


While independence is a useful property, we do not often encounter two independent events.  
A more common situation is when two events $X$ and $Y$ are independent given an additional event $Z$, denoted as 
 \begin{equation}
     X \bot Y | Z
 \end{equation}
A distribution $P$ satisfies $X \bot Y | Z$ if and only if $P(X,Y | Z)=P(X | Z)P(Y | Z)$. 

Note that, similar calculation as in Examples~\ref{example:independenceexample} and \ref{example:noindependence} can be done to work with conditional independence, with the only changes on replacing the computation of marginal probabilities $P(X)$ and $P(Y)$ with the computation of conditional probabilities $P(X | Z)$ and $P(Y | Z)$.


\section{Querying Joint Probability Distributions}

A joint distribution $P(X_1,...,X_n)$ contains an exponential number $2^n$ of real probability values and it can be hard to make sense of them. It is desirable that we are able to infer useful information by making queries. 
%
In the following, we introduce a few categories of queries. 

\subsection*{Probability queries} 

are to compute distribution of a subset of random variables, given  evidence (i.e., the values) of another subset of random variables. Formally, it is to compute
\begin{equation}
    P(X_{i1},...,X_{ik}| X_{j1}=x_{j1},...,X_{jl}=x_{jl})
\end{equation}
where $x_{j1},...,x_{jl}$ are evidence for the random variables $X_{j1},...,X_{jl}$, respectively. Moreover, we have  $\{X_{i1},...,X_{ik}\}\cup \{X_{j1},...,X_{jl}\}\subseteq \{X_1,...,X_n\}$ and $\{X_{i1},...,X_{ik}\}\cap  \{X_{j1},...,X_{jl}\} = \emptyset$. 

\subsection*{Maximum a posteriori (MAP) queries}

In addition to probability queries which concern the (conditional) probability of the occurrence of events, we may be interested in MAP-style queries, which is to find a joint assignment to some subset of variables that has the highest probability. For simplicity, we let $\{X_1,...,X_k\}, \{X_{k+1},...,X_l\}, \{X_{l+1},...,X_n\}$ be three disjoint subsets of the random variables $\{X_1,...,X_n\}$. Then, the MAP query is of the form  
\begin{equation}
\begin{array}{rl}
     &  MAP(X_1,...,X_k|X_{l+1},...,X_n) \\
   =  & \displaystyle\argmax_{x_1,...,x_k} \sum_{x_{k+1},...,x_{l}}P(X_{1}=x_1,...,X_{l}=x_l| X_{l+1}=x_{l+1},...,X_{n}=x_{n})
\end{array}
\end{equation}
Intuitively, we have the evidence for variables $\{X_{l+1},...,X_n\}$, and intend to find the joint assignment for $\{X_1,...,X_k\}$. Because we do not have information about $\{X_{k+1},...,X_l\}$, we marginalise them. 

\begin{example}
Consider a probability table for variable $X_1$ as in Table~\ref{tab:tableforx1}.  
\begin{table}[!htbp]
    \centering
    \begin{tabular}{|c|c|}
    \hline
        $X_1 = 0$ & $X_1 = 1$ \\
        \hline
        0.4 & 0.6\\
        \hline
    \end{tabular}
    \caption{Probability table for $X_1$}
    \label{tab:tableforx1}
\end{table}

Then, we have $MAP(X_1)=1$. 
\end{example}

\begin{example}
Consider a joint probability table for variables $X_1$ and $X_2$ as in Table~\ref{tab:tableforx1x2}. 
\begin{table}[!htbp]
    \centering
    \begin{tabular}{|c|c|c|}
    \hline
        $X_1$ & $X_2$ & $P(X_1,X_2)$ \\
        \hline
        0 & 0 & 0.04\\
        0 & 1 & 0.36 \\
        1 & 0 & 0.3 \\
        1 & 1 & 0.3\\
        \hline
    \end{tabular}
    \caption{Joint probability table for $X_1$ and $X_2$}
    \label{tab:tableforx1x2}
\end{table}

Then, we have $MAP(X_2)=1$. 
\end{example}

%Maximum a posteriori probability 
%Also called MPE (Most Probable Explanation) 
%What is 
%Marginal MAP Queries


