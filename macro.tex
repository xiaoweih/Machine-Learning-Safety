% -*- coding: utf-8 -*-
\newcommand{\dnnfunction}{\phi}
\newcommand{\rnnfunction}{\psi}
\newcommand{\covered}[3]{{#1}_{#2}#3}
\newcommand{\valuefunction}{g}
\newcommand{\featureactivation}{a}
\newcommand{\neuronpair}{\alpha}
\newcommand{\feature}{\psi}
\newcommand{\features}{\Psi}
\newcommand{\neuronpairs}{\mathsf{O}}
\newcommand{\testsuite}{\mathcal{T}}
\newcommand\testsuites{\ensuremath{\mathcal{T}}\xspace}
\newcommand{\distance}[2]{ ||#1||_{#2}}
\newcommand{\powerset}[1]{{\cal P}(#1)}
\newcommand{\real}{\mathbb{R}}
\newcommand{\nat}{\mathbb{N}}

\newcommand{\sequence}[1]{\mathtt{#1}}
\newcommand{\metric}{M}
\newcommand{\network}{{\cal N}\xspace}
\newcommand{\requirement}{{\cal R}}
\newcommand{\assertmethod}{{\cal A}}

\newcommand{\sigmoid}{\sigma}
\DeclareMathOperator*{\argmax}{arg\,max}
\DeclareMathOperator*{\argmin}{arg\,min}



\usepackage{xifthen}% provides \isempty test
\newcommand\lambdaone[1]{\ensuremath{\ifthenelse{\isempty{#1}}{}{(#1)}}}
\newcommand\lambdatwo[2]{\ensuremath{\ifthenelse{\isempty{#1#2}}{}{(#1, #2)}}}
% \newcommand{\Dom}[1]{\ensuremath{\operatorname{Dom}\lambdaone{#1}}\xspace}
% \newcommand{\Codom}[1]{\ensuremath{\operatorname{Codom}\lambdaone{#1}}\xspace}
\newcommand{\Dom}[1]{\ensuremath{\mathsf{Dom}\lambdaone{#1}}\xspace}
\newcommand{\Codom}[1]{\ensuremath{\mathsf{Codom}\lambdaone{#1}}\xspace}
% \newcommand{\Dom}[1]{\ensuremath{\mathsf{Dom}_{#1}}\xspace}
% \newcommand{\Codom}[1]{\ensuremath{\mathsf{Codom}_{#1}}\xspace}

\newcommand{\inputdomain}{\ensuremath{\DD_X}\xspace}
\newcommand{\outputdomain}{\ensuremath{\DD_Y}\xspace}
\newcommand\layerdom[1]{\ensuremath{\LL_{#1}}\xspace}
\newcommand\featdom[1]{\ensuremath{\FF_{#1}}\xspace}
\newcommand\featcomp[2]{\ensuremath{\FF_{{#1},{#2}}}\xspace}

%\newtheorem{theorem}{Theorem}
%\newtheorem{definition}{Definition}

\newcommand{\param}[1]{\texttt{#1}}
% \newcommand{\num}[2]{$#1$$,$$#2$}

\newcommand{\xiaowei}[1]{{\color{blue}#1}}
\newcommand{\sven}[1]{{\color{blue}sven: #1}}
\newcommand{\gaojie}[1]{{\color{green}#1}}

% Custom inline notes
% \newcommand{\note}[2][]{\todo[color=green!20,caption={rem}, #1]{%
%     #2}}
\newcommand{\inote}[2][]{\todo[inline,color=green!20,caption={2do}, #1]{%
    \begin{minipage}{\textwidth-4pt}#2\end{minipage}}}
\newcommand{\itodo}[2][]{\todo[inline,caption={2do}, #1]{%
    \begin{minipage}{\textwidth-4pt}#2\end{minipage}}}

\addtolength\marginparwidth{1.1cm}
\newcommand{\nb}[2][]{\todo[color=green!20,caption={rem}, #1]{%
    NB: #2}}
\newcommand{\nbtodo}[2][]{\todo[#1]{%
    NB: #2}}
\newcommand{\js}[2][]{\todo[color=blue!20,caption={rem}, #1]{%
    JS: #2}}

\newcommand{\red}[1]{{\leavevmode\color{red}#1}\xspace}%
\newcommand{\nberth}[1]{{\leavevmode\color{green!50!black}#1}\xspace}%
\newcommand{\removable}[1]{{\leavevmode\color{teal}#1}\xspace}%
\newenvironment{redenv}{\color{red}}{}
\newenvironment{greenenv}{\color{green}}{}

\newcommand{\adhoc}		{{\itshape	 ad hoc\/}\xspace}
\newcommand{\afortiori}		{{\itshape   a fortiori\/}\xspace}
\newcommand{\aka}		{{\itshape	    aka\/}\xspace}
\newcommand{\aposter}		{{\itshape a posteriori\/}\xspace}
\newcommand{\apriori}		{{\itshape     a priori\/}\xspace}
\newcommand{\cf}		{{\itshape	      cf.}\xspace}
\newcommand{\vs}		{{\itshape	       vs}\xspace}
\newcommand{\defacto}		{{\itshape     de facto\/}\xspace}
\newcommand{\eg}		{{\itshape	    e.g.,}\xspace}
\newcommand{\idem}		{{\itshape	   idem\/}\xspace}
\newcommand{\modulo}		{{\itshape	 modulo\/}\xspace}
\newcommand{\ie}		{{\itshape	    i.e.,}\xspace}
\newcommand{\perse}		{{\itshape	   per se}\xspace}
\newcommand{\wrt}		{{\itshape	   w.r.t.}\xspace}
\newcommand{\st}		{{\itshape	    s.t\/}\xspace}

\newcommand{\DeepConcolic}{\textsf{DeepConcolic}\xspace}
\newcommand{\TestRNN}{\textsf{TestRNN}\xspace}
\newcommand{\EKiML}{\textsf{EKiML}\xspace}
\newcommand{\GUAP}{\textsf{GUAP}\xspace}

\newcommand\deepconcolic{\DeepConcolic}
\newcommand\testRNN{\TestRNN}
\newcommand\testrnn{\TestRNN}
\newcommand\ekiml{\EKiML}

\newcommand\dbnabstr{\texttt{dbnabstr}\xspace}

\renewcommand\L{\ensuremath{\mathcal L}\xspace}%
\renewcommand\P{\ensuremath{\mathcal P}\xspace}%
\newcommand\A{\ensuremath{\mathcal A}\xspace}%
\newcommand\B{\ensuremath{\mathcal B}\xspace}%
\newcommand\C{\ensuremath{\mathcal C}\xspace}%
%\newcommand\D{\ensuremath{\mathcal D}\xspace}%
\newcommand\F{\ensuremath{\mathcal F}\xspace}%
\newcommand\G{\ensuremath{\mathcal G}\xspace}%
%\newcommand\I{\ensuremath{\mathcal I}\xspace}%
\newcommand\K{\ensuremath{\mathcal K}\xspace}%
\newcommand\N{\ensuremath{\mathcal N}\xspace}%
\newcommand\Q{\ensuremath{\mathcal Q}\xspace}%
\newcommand\R{\ensuremath{\mathcal R}\xspace}%
\newcommand\T{\ensuremath{\mathcal T}\xspace}%
\newcommand\Z{\ensuremath{\mathcal Z}\xspace}%
\newcommand\BB{\ensuremath{\mathbb B}\xspace}%
\newcommand\DD{\ensuremath{\mathbb D}\xspace}%
\newcommand\FF{\ensuremath{\mathbb F}\xspace}%
\newcommand\GG{\ensuremath{\mathbb G}\xspace}%
\newcommand\HH{\ensuremath{\mathbb H}\xspace}%
\newcommand\LL{\ensuremath{\mathbb L}\xspace}%
\newcommand\RR{\ensuremath{\mathbb R}\xspace}%
\newcommand\Ls{\ensuremath{\mathscr L}\xspace}%


\newcommand{\context}{{\cal C}}
\newcommand{\contexts}{{\tt C}}
\newcommand\Layers{\ensuremath{\mathsf{Layers}}\xspace}
\newcommand\layer[1]{\ensuremath{l_{#1}}\xspace}
\newcommand{\networks}{{\tt N}}
\newcommand\Features[1]{\ensuremath{Λ% \mathsf{Features}
    _{#1}}\xspace}

\newcommand{\dist}{{\cal D}}

\newcommand{\dimension}{h}
\newcommand\lowdim[1]{\ensuremath{|Λ_{#1}|}\xspace}

% \newcommand\DiscrFeatsSpace[1]{\ensuremath{\L^{♯}_{#1}}\xspace}
\newcommand\DiscrFeatsSpace[1]{\ensuremath{\FF^{♯}_{#1}}\xspace}
% \newcommand\DiscrFeatsElt[1]{\ensuremath{\Ls^{♯}_{#1}}\xspace}
\newcommand\DiscrFeatsElt[1]{\ensuremath{F^{♯}_{#1}}\xspace}
\newcommand\AllSubFeatSpaces[1]{\ensuremath{\FF^{♯}_{#1}}\xspace}

% \newcommand\Subfeatspaces[2]{\ensuremath{Λ^{♯}_{#1,#2}}\xspace}
\newcommand\Subfeatspaces[2]{\ensuremath{\FF^{♯}_{#1,#2}}\xspace}
\newcommand\subfeatspace[3]{\ensuremath{f^{♯#3}_{#1,#2}}\xspace}
\newcommand\subfeatlb[3]{\ensuremath{f^{[#3}_{#1,#2}}\xspace}
\newcommand\subfeatub[3]{\ensuremath{f^{#3[}_{#1,#2}}\xspace}
% \newcommand\featsubsp[2]{\ensuremath{ℓ^{♯}_{#1,#2}}\xspace}
\newcommand\featsubsp[2]{\ensuremath{\mathsf{Discr}^{♯}_{#1,#2}}\xspace}
% \newcommand\subfeatlb[3]{\ensuremath{\left\lfloor\subfeatspace#1#2#3\right\rfloor}\xspace}
% \newcommand\subfeatlb[3]{\ensuremath{\subfeatspace#1#2{#3[}}\xspace}

\renewcommand\P[1]{\ensuremath{\mathrm{Pr}\!\left(#1\right)}\xspace}

% TMP: latent/hidden features
\newcommand\LF[1][h]{#1idden feature\xspace}
\newcommand\LFs[1][h]{#1idden features\xspace}
\newcommand\LFI[1][h]{#1idden feature interval\xspace}
\newcommand\LFIs[1][h]{#1idden feature intervals\xspace}
\newcommand\HF[1][h]{#1idden feature\xspace}
\newcommand\HFs[1][h]{#1idden features\xspace}
\newcommand\HFI[1][h]{#1idden feature interval\xspace}
\newcommand\HFIs[1][h]{#1idden feature intervals\xspace}


% from BN paper:
\newcommand\BNaStructure[1]{\ensuremath{\B_{#1}}\xspace}%
\newcommand\BNa[2]{\ensuremath{\B_{#1,#2}}\xspace}%
\let\BN=\BNa
\newcommand\BNode[2]{\ensuremath{⦇f^{♯}_{#1,#2}⦈}\xspace}%
\newcommand\IPr[2]{\ensuremath{\mathcal{IP}\!({#2})}\xspace}%
\newcommand\CPr[4]{\ensuremath{\mathcal{CP\!}_{#1}\!({#3}|#4)}\xspace}%
\newcommand\MPr[2]{\ensuremath{\mathcal{P}\!_{#1}{(#2)}}\xspace}%
\newcommand\BNFCov[2]{\ensuremath{\mathrm{BFCov}(\BNa{#1}{#2})}\xspace}%
\newcommand\BNFdCov[2]{\ensuremath{\mathrm{BFdCov}(\BNa{#1}{#2})}\xspace}%
\newcommand\BNFxCov[2]{\ensuremath{\mathrm{BFxCov}(\BNa{#1}{#2})}\xspace}%


\newcommand\Nms{\ensuremath{\N_{\mathsf{ms}}}\xspace}%

%\newcommand{\realnumber}{\mathcal{R}}
\newcommand{\labels}{\mathcal{L}}

%
\newcommand\Nmx{\ensuremath{\N_{\mathsf{mx}}}\xspace}%

\newcommand\Minimise[2][]{\ensuremath{\mathit{Minimise}(#2)}\xspace}%
\newcommand\Target[1]{\ensuremath{\mathit{Target}(#1)}\xspace}%
\newcommand\Constr[2][]{\ensuremath{\mathit{Constr}_{#1}(#2)}\xspace}%
\newcommand\Replic[2]{\ensuremath{\mathit{Replic}_{#1}(#2)}\xspace}%
\newcommand\LInf[2]{\ensuremath{\|#1 - #2\|_{∞}}\xspace}%

% ---



% tuto part:

\newcommand\Xtrain{\ensuremath{X_{\mathit{train}}}\xspace}%
\newcommand\Ytrain{\ensuremath{Y_{\mathit{train}}}\xspace}%
\newcommand\Xtest{\ensuremath{X}\xspace}%
\newcommand\Ytest{\ensuremath{Y}\xspace}%

\newcommand\dthr{\ensuremath{d_{\mathrm{thr}}}\xspace}%
\newcommand\dminHard{\ensuremath{d_{\min}}\xspace}%
\newcommand\dminNoise{\ensuremath{d_{\min}^+}\xspace}%

% \newcommand\Nfm{\ensuremath{\N_{\mathsf{FM}}}\xspace}%
% \newcommand\Nfm{\lstinline[language=sh,basicstyle=\ttfamily\small]%
%   {fashion\_mnist\_medium.h5}\xspace}%
% \newcommand\Nhar{\ensuremath{\N_{\mathsf{HAR}}}\xspace}%
\newcommand\Nhs{\lstinline[language=sh,basicstyle=\ttfamily\small]%
  {har_dense.h5}\xspace}%
\newcommand\Nhl{\lstinline[language=sh,basicstyle=\ttfamily\small]%
  {har_conv2d.h5}\xspace}%

\definecolor{codegreen}{rgb}{0,0.6,0}
\definecolor{codegray}{rgb}{0.5,0.5,0.5}
\definecolor{codepurple}{rgb}{0.58,0,0.82}
\definecolor{backcolour}{rgb}{0.95,0.95,0.92}

\lstdefinestyle{mystyle}{
    backgroundcolor=\color{backcolour},   
    commentstyle=\color{codegreen},
    keywordstyle=\color{magenta},
    numberstyle=\tiny\color{codegray},
    stringstyle=\color{codepurple},
    basicstyle=\ttfamily\footnotesize,
    breakatwhitespace=false,         
    breaklines=true,                 
    captionpos=b,                    
    keepspaces=true,                 
    numbers=left,                    
    numbersep=5pt,                  
    showspaces=false,                
    showstringspaces=false,
    showtabs=false,                  
    tabsize=2
}

\lstset{style=mystyle}

\newcommand{\functionname}[1]{\textbf{\textit{#1}}}
\newcommand{\humanoracle}{h}

%\newtheorem{definition}{Definition}
%\newtheorem{example}{Example}
%\newtheorem{theorem}{Theorem}
\newtheorem{newquestion}{Question}
%\newtheorem*{answer*}{Answer}
\newtheorem{newanswer}{Answer}


%\DeclareMathOperator*{\argmax}{argmax} % thin space, limits underneath in displays
%\DeclareMathOperator*{\argmin}{argmin} % thin space, limits underneath in displays

\makeatletter
\newcommand{\xMapsto}[2][]{\ext@arrow 0599{\Mapstofill@}{#1}{#2}}
\def\Mapstofill@{\arrowfill@{\Mapstochar\Relbar}\Relbar\Rightarrow}
\makeatother

\newcommand{\dsep}{\text{d-sep}}


\DeclareRobustCommand{\bbone}{\text{\usefont{U}{bbold}{m}{n}1}}
\DeclareMathOperator{\EX}{\mathbb{E}}


\newcommand{\pre}{pre}
\newcommand{\con}{con}


\newcommand{\layers}{\mathbb{S}}
\newcommand{\layerConnections}{\mathbb{T}}
\newcommand{\domains}{D}


\usepackage{stackengine}
\def\defequal{\mathrel{\ensurestackMath{\stackon[1pt]{=}{\scriptscriptstyle\Delta}}}}


\newboolean{notincludedinbook}
\setboolean{notincludedinbook}{true}
\ifthenelse{\boolean{notincludedinbook}}
{% True case
    \newcommand{\notincludedinbook}[1]{}
}
  {% false case
   \newcommand{\notincludedinbook}[1]{#1}%
  }


\newboolean{noexercises}
\setboolean{noexercises}{true}
\ifthenelse{\boolean{noexercises}}
{% True case
    \newcommand{\noexercises}[1]{}
}
  {% false case
   \newcommand{\noexercises}[1]{#1}%
  }


\newcommand{\setofcoveringmethods}{\mathsf{F}}
\newcommand{\coveringmethod}{cov}
\newcommand{\testobjectives}{obj}
\newcommand{\requirements}{\mathcal{R}}

\newcommand{\realnumber}{\mathbb{R}}

\newcommand{\eventually}{\Diamond}

%\newtheorem{lemma}{Lemma}

% Local variables:
% TeX-master: "main"
% End:
