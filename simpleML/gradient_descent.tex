\newpage
\chapter{Loss Function and Gradient Descent}

Before proceeding to deep learning, we use this section to discuss two key concepts: loss function and gradient descent. Technically, machine learning is to optimise a certain loss function over a set of training instances. A carefully designed loss function can significantly improve the performance of the trained model. A recent trend in machine learning research -- as we will explain later in Part~\ref{chap:verification} -- also designs loss functions to integrate safety properties. 
Once the loss function is designed, it is of our interest to find an optimal solution (i.e., a trained model) with respect to the loss function. However, the optimal solution might not be easily achievable, in particular for high-dimensional problems and/or for  machine learning models with many parameters. 
%Most machine learning algorithms involve optimisation, and 
In this case, 
gradient descent (and its variant such as stochastic gradient descent) is an effective method to find sub-optimal, yet often satisfactory, solutions. This chapter introduces the basic ideas of both loss function and gradient descent, and explains how they are applied to the case of linear regression. 

\section{Loss Functions}

This section introduces variants of loss functions, which are learning objectives. %
Assume that, we have a model $f_{\textbf{W}}$, being it a model whose parameters are just initialised or a model who appears during the training process. The dataset is $D=\{(\textbf{x}_i,y_i)~|~i\in \{1..n\}\}$ is a labelled dataset. We are considering the classification task. 

In previous sections, we have introduced mean squared error (MSE), repeated as below, for linear regression. MSE is one of the most widely used loss functions. 

%\subsection*{Mean Squared Error (MSE)}

\begin{equation}\label{equ:mseloss}
    \hat{L}(f_\textbf{W}) = \frac{1}{m}\sum_{i=1}^m(f_{\textbf{W}   }(\textbf{x}^{(i)})-y^{(i)})^2
\end{equation}
Intuitively, MSE measures the average areas of the square created by the predicted and ground-truth points. 

\subsection*{Mean Absolute Error}

\begin{equation}\label{equ:maeloss}
    \hat{L}(f_\textbf{w}) = \frac{1}{m}\sum_{i=1}^m|f_{\textbf{W}}(\textbf{x}^{(i)})-y^{(i)}|
\end{equation}
Unlike MSE which concerns the areas of square, MAE concerns the geometrical distance between 
the predicted and ground-truth points. Comparing to MSE whose derivative can be easily computed, it is harder to compute derivative for MAE. 

\subsection*{Root Mean Squared Error (RMSE)}

RMSE is very similar to MSE, except for the square root operation. 

\begin{equation}\label{equ:rmseloss}
    \hat{L}(f_\textbf{w}) = \sqrt{\frac{1}{m}\sum_{i=1}^m(f_{\textbf{W}}(\textbf{x}^{(i)})-y^{(i)})^2}
\end{equation}

\subsection*{Binary Cross Entropy Cost Function}

When considering binary classification, i.e., $C=\{0,1\}$, we may utilise information theoretical concepts, cross entropy, which measures the difference between two distributions for the predictions and the ground truths.  
%
\begin{equation}\label{equ:binarycrossentropyloss}
    \hat{L}(f_\textbf{w}) = \sum_{i=1}^m - y^{(i)}\log  f_{\textbf{W}}(\textbf{x}^{(i)})-  (1-y^{(i)})\log  (1-f_{\textbf{W}}(\textbf{x}^{(i)}))
\end{equation}
%
where 
Cross entropy loss works better after the softmax layer, because the output of the softmax layer represents a distribution. 


\subsection*{Categorical Cross Entropy Cost Function}

Extending the above to multiple classes, we may have 
\begin{equation}\label{equ:crossentropyloss}
    \hat{L}(f_\textbf{w}) = - \sum_{i=1}^m \sum_{c\in C} \textbf{y}^{(i)}_c\log  [f_{\textbf{W}}(\textbf{x}^{(i)})]_c
\end{equation}
where $\textbf{y}^{(i)}$ is the one-hot representation of the ground truth $y^{(i)}$ and $\textbf{y}^{(i)}_c$ denotes the component of $\textbf{y}^{(i)}$ that is for the class $c$. Also, unlike the previous notations, $f_{\textbf{W}}(\textbf{x}^{(i)})$ is a probability distribution   of the prediction over $\textbf{x}^{(i)}$, and $[f_{\textbf{W}}(\textbf{x}^{(i)})]_c$ denotes the component of $f_{\textbf{W}}(\textbf{x}^{(i)})$ that is for the class $c$. 



\section{Gradient Descent}

\subsection*{Derivative of a function}

Given a function $\hat{L}({x})$, its derivative is the slope of $\hat{L}({x})$ at point ${x}$, written as $\hat{L}'({x})$. It specifies how to scale a small change in input to obtain a corresponding change in the output. Specifically, 
\begin{equation}
    \hat{L}({x}+\epsilon) \approx \hat{L}({x}) + \epsilon \hat{L}'({x})
\end{equation}

Moreover, define the sign function as the following: 
\begin{equation}
    sign(x) = 
    \begin{cases}
    -1 & \text{if }x < 0 \\
    0 & \text{if }x = 0 \\
    1 & \text{otherwise} 
    \end{cases}
\end{equation}
Then, we have that 
\begin{equation}
    \hat{L}({x}-\epsilon sign(\hat{L}'({x}))) < \hat{L}({x})
\end{equation}
Therefore, we can reduce $\hat{L}({x})$  by moving ${x}$ in small steps with an opposite sign of derivative.  

\subsection*{Gradient}

Gradient generalizes notion of derivative where derivative is with respect to a vector. 
\begin{equation}
    \nabla_{\textbf{x}}(\hat{L}(\textbf{x})) = (\frac{{\partial \hat{L}(\textbf{x})}}{{\partial x_1}},...,\frac{{\partial \hat{L}(\textbf{x})}}{{\partial x_n}})
\end{equation}
where the partial derivative $\displaystyle \frac{{\partial \hat{L}(\textbf{x})}}{{\partial x_i}}$ measures how the function $\hat{L}$ changes when only variable $x_i$ increases at point $\textbf{x}$.  

\subsection*{Critical points}  are where every element of the gradient is equal to zero, i.e., 
\begin{equation}
    \nabla_{\textbf{x}}(\hat{L}(\textbf{x})) = 0 \equiv 
    \begin{cases}
    \displaystyle\frac{{\partial \hat{L}(\textbf{x})}}{{\partial x_1}} = 0 \\
    ... \\
    \displaystyle\frac{{\partial \hat{L}(\textbf{x})}}{{\partial x_n}} = 0\\
    \end{cases}
\end{equation}


\subsection*{Gradient descent on linear regression}

Given Equation (\ref{equ:linearregression2}), we have that 
\begin{equation}\label{equ:gradientlinearregression}
    \begin{array}{cl}
         & \nabla_{\textbf{w}}\hat{L}(f_{\textbf{w}}) \\
        = &  \nabla_{\textbf{w}}\frac{1}{m} ||\textbf{X}\textbf{w} - \textbf{y}||_2^2 \\
        = & \nabla_{\textbf{w}}[(\textbf{X}\textbf{w} - \textbf{y})^T(\textbf{X}\textbf{w} - \textbf{y})]\\
        = & \nabla_{\textbf{w}}[\textbf{w}^T\textbf{X}^T\textbf{X}\textbf{w}-2\textbf{w}^T\textbf{X}^T\textbf{y}+\textbf{y}^T\textbf{y}]\\
         = & 2 \textbf{X}^T\textbf{X}\textbf{w} - 2 \textbf{X}^T\textbf{y}
    \end{array}
\end{equation}
Therefore, we can follow the following gradient descent algorithm to solve linear regression: 
\begin{enumerate}
    \item Set step size $\epsilon$ and tolerance $\delta$ to small positive numbers
    \item While $|| \textbf{X}^T\textbf{X}\textbf{w} -  \textbf{X}^T\textbf{y}||_2 > \delta$ do \begin{equation}
        \textbf{x} \leftarrow \textbf{x} - \epsilon(\textbf{X}^T\textbf{X}\textbf{w} -  \textbf{X}^T\textbf{y})
    \end{equation}
    \item Return $\textbf{x}$ as a solution
\end{enumerate}


\subsection*{Analytical solution on linear regression}

We may be able to avoid iterative algorithm and jump to the critical point by solving the following equation for $\textbf{x}$: 
\begin{equation}
    \nabla_{\textbf{w}}\hat{L}(f_{\textbf{w}}) = 0
\end{equation}
By Equation (\ref{equ:gradientlinearregression}), we have that 
\begin{equation}
     \textbf{X}^T\textbf{X}\textbf{w} -  \textbf{X}^T\textbf{y} = 0
\end{equation}
That is, 
\begin{equation}
    \textbf{w} = (\textbf{X}^T\textbf{X})^{-1}\textbf{X}^T\textbf{y}
\end{equation}